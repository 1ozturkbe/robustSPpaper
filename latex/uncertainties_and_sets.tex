\section{Uncertainties and Sets}
\label{uncertainties_and_sets}

As aforementioned in Section~\ref{sec:robustvsstochastic}, one of the advantages
of \gls{ro} over \gls{so} is the fact that it is more effective in absence of
data, since the problem has uncertainty set bounds on parameters as inputs instead
of complete probability distributions.
These uncertainties, given by three times the \gls{cv}\footnote{The \gls{cv}
is defined as follows: $\text{CV} = \frac{\sigma}{|\mu|}$, where $\sigma$ is the standard deviation and $\mu$ is the mean of the parameter.},
are listed in Table~\ref{tab:uncertainties}. Since for the rest of this work
all standard deviations ($\sigma$) are normalized by the means of the parameters, we will use $3\sigma$
to represent $3\text{CV}$.

\begin{table}
\begin{center}
\caption{\label{tab:uncertainties} Parameters and Uncertainties (increasing order)}
\begin{tabular}{c c c c c}
\hline
Parameters & Description & Value & \% Uncert. ($3\sigma$) \\
\hline
$S_{\rm{wetratio}}$ & wetted area ratio & 2.075 & 3\\
e & span efficiency & 0.92 & 3\\
$\mu$ & air viscosity (SL) & $1.78 \times 10^{-5}~\mathrm{kg/(ms)}$ & 4 \\
$\rho$ & air density (SL) & 1.23 $\mathrm{kg/m^3}$ & 5 \\
$C_{L_{\rm{max}}}$ & stall lift coefficient & 1.6 & 5\\
k & fuselage form factor & 1.17 & 10\\
$C_{f_{\rm{fuse, ref}}}$ & fuselage skin friction factor & 0.455 & 10 \\
$\rho_{\rm{p}}$ & payload density & 1.5 $\mathrm{kg/m^3}$ & 10 \\
$\tau$ & airfoil thickness ratio & 0.12 & 10\\
$N_{\rm{ult}}$ & ultimate load factor & 3.3 & 15\\
$V_{\rm{min}}$ & takeoff speed & 30 m/s & 20\\
$W_{\rm{p}}$ & payload weight & 6250 N & 20\\
$W_{\rm{w}_{\rm{coeff,strc}}}$ & wing structural weight coefficient & $2 \times 10^{-5}~1/\mathrm{m}$ & 20\\
$W_{\rm{w}_{\rm{coeff,surf}}}$ & wing surface weight coefficient & 60 $\mathrm{N/m^2}$ & 20\\
\hline
\end{tabular}
\end{center}
\end{table}

In this case of a conceptual aircraft design with no prior data,
the parameter uncertainties reflect aerospace engineering intuition.
The wing weight coefficients $W_{\rm{w}_{\rm{coeff,strc}}}$ and $W_{\rm{w}_{\rm{coeff,surf}}}$,
and the ultimate load factor $N_{\rm{ult}}$ have
large $3\sigma$s because the build quality of aircraft components is
often difficult to quantify with a large degree of certainty.
The payload weight and density ($W_{\rm{p}}$ and $\rho_{\rm{p}}$) have large uncertainties for similar reasons,
since the payload is often developed concurrently with the aircraft.
Parameters that engineers take to be
physical constants (sea level air viscosity and density, $\mu$ and $\rho$) and those that can be determined or manufactured with a relatively
high degree of accuracy ($S_{\rm{wetratio}}$, $e$) have relatively low deviations.
Parameters that require testing to determine ($C_{L_{\rm{max}}}$, $C_{f_{\rm{fuse, ref}}}$,
$V_{\rm{min}}$) have a level of uncertainty
that reflects the expected variance of empirical studies. However, note that
these quantities are ultimately picked by the designer using prior experience and data,
and the level of conservatism in the
design will be greatly affected by the chosen $3\sigma$s.

\section{Uncertainties and Sets}

The uncertainties for the different constants in the problem have been determined
considering the parameters in aircraft design that often have the largest uncertainty.
These uncertainties are listed in Table~\ref{tab:uncertainties}.

\begin{table}
\begin{center}
\caption{\label{tab:uncertainties} Constants and Uncertainties (increasing order)}
\begin{tabular}{c c c c}
\hline
Constant & Description & Value & \% Uncert. ($3\sigma$) \\
$S_{wetratio}$ & wetted area ratio & 2.075 & 3\\
e & span efficiency & 0.92 & 3\\
$\mu$ & viscosity of air & 1.775e-5 $kg/(ms)$ & 4 \\
$\rho$ & air density & 1.23 $kg/m^3$ & 5 \\
$C_{L_{max}}$ & stall lift coefficient & 1.6 & 5\\
k & fuselage form factor & 1.17 & 10\\
$\tau$ & airfoil thickness ratio & 0.12 & 10\\
$N_{ult}$ & ultimate load factor & 3.3 & 15\\
$V_{min}$ & takeoff speed & 25 $m/s$ & 20\\
$W_0$ & payload weight & 6250 $N$ & 20\\
$W_{w_{coeff1}}$ & wing weight coefficient 1 & 2e-5 $1/m$ & 20\\
$W_{w_{coeff2}}$ & wing weight coefficient 2 & 60 $N/m^2$ & 20\\
\hline
\end{tabular}
\end{center}
\end{table}

The parameter uncertainties reflect aerospace engineering intuition.
The wing weight coefficients $W_{w_{coeff1}}$ and $W_{w_{coeff2}}$, and the ultimate load factor $N_{ult}$ have
large $3\sigma$s because build quality of aircraft components often difficult to quantify with a large degree of certainty.
The payload weight ($W_0$) has a large uncertainty, because it is valuable if the aircraft
has the flexibility to accommodate larger payloads. Parameters that engineers take to be
physical constants ($\mu$, $\rho$) and those that can be determined/manufactured with a relatively
high degree of accuracy ($S_{wetratio}$, $e$) have relatively low deviations.
Parameters that require testing to determine ($C_{L_{max}}$, $V_{min}$) have a level of uncertainty
that reflects the expected variance of the parameters.

\section*{Appendix}

\subsection{Robust Linear Programming: A Quick Review} \label{LP_to_GP}

As mentioned earlier, principles from robust linear programming are used
formulate an approximate robust geometric program.\\[12pt]
Consider the system of linear constraints
\begin{equation*}
    \mathbb{A}\mathbf{x} + \mathbf{b} \leq 0
\end{equation*}
where
\begin{align*}
    \mathbb{A} &\text{ is $m \times n$}\\
    \mathbf{x} &\text{ is $n \times 1$}\\
    \mathbf{b} &\text{ is $m \times 1$}\\
\end{align*}
where the uncertain data is contained in a set defined by equations \eqref{Data} and \eqref{perturbation_set}.

\subsubsection{Box Uncertainty Set}
If the perturbation set $\mathcal{Z}$ given in equation \eqref{perturbation_set} is a box
uncertainty set, i.e. $\|\mathbf{\zeta}\|_{\infty} \leq \Gamma$, then the robust formulation of the $i$th constraint is equivalent to
\begin{equation}
    \Gamma \textstyle{\sum}_{l=1}^L |- {b}^l_{i} - \mathbf{a}^l_i\mathbf{x}| + \mathbf{a}^0_i\mathbf{x} + b^0_i \leq 0
    \label{box_absolute}
\end{equation}
If only $b$ is uncertain, i.e. $A^l = 0,~\forall l = 1,2,...,L$, then equation \eqref{box_absolute} becomes
\begin{equation}
    \textstyle{\sum}_{l=1}^L \mathbf{a}^0_{i}\mathbf{x} + b^0_{i} + \Gamma \textstyle{\sum}_{l=1}^L |b^l_{i}| \leq 0
    \label{box_coeff}
\end{equation}
which is a linear constraint.

On the other hand, if $A$ is uncertain, the equation \eqref{box_absolute} is equivalent to the following set of linear constraints
\begin{equation}
    \label{box_linear}
    \begin{split}
        \Gamma \textstyle{\sum}_{l=1}^L w^l_{i} + \mathbf{a}^0_{i}\mathbf{x} + b^0_{i} &\leq 0 \\
        - b^l_{i} - \mathbf{a}^l_{i}\mathbf{x} &\leq w^l_{i},~\forall l \in 1,...,L\\
        b^l_{i} + \mathbf{a}^l_{i}\mathbf{x} &\leq w^l_{i},~\forall l \in 1,...,L\\
    \end{split}
\end{equation}

\subsubsection{Elliptical Uncertainty Set}
If the perturbation set $\mathcal{Z}$ is an elliptical, i.e. $\textstyle{\sum}_{l=1}^L\frac{\zeta_l^2}{\sigma_l^2} \leq \Gamma^2$,
then the robust formulation of the $i^{th}$ constraint is equivalent to
\begin{equation}
    \Gamma \sqrt{\textstyle{\sum}_{l=1}^L \sigma_l^2(- b^l_{i} - \mathbf{a}^l_{i}\mathbf{x})^2} + \mathbf{a}^0_{i}\mathbf{x} + b^0_{i} \leq 0
    \label{ell_absolute}
\end{equation}
which is a second order conic constraint.

If only $b$ is uncertain, i.e. $\mathbb{A}^l = 0,~\forall l = 1,2,...,L$, then equation \eqref{ell_absolute} becomes
\begin{equation}
    \textstyle{\sum}_{l=1}^L \mathbf{a}^0_{i}\mathbf{x} + b^0_{i} + \Gamma \sqrt{\textstyle{\sum}_{l=1}^L \sigma_l^2(b^l_{i})^2} \leq 0
    \label{ell_coeff}
\end{equation}
which is a linear constraint.

\subsubsection{Norm-1 Uncertainty Sets}

If the perturbation set represented by $\mathcal{Z}$ is a norm-1 uncertainty set, i.e. $\|\mathbf{\zeta}\|_1 \leq \Gamma$,
then the robust constraint is
\begin{equation}
    \textstyle{\sum}_{l=1}^L \mathbf{a}^0_{i}\mathbf{x} + b^0_{i} + \Gamma \max_{l=1,..,L} |b^l_{i}| \leq 0
    \label{rom_coeff}
\end{equation}
when $\mathbb{A}^l = 0$, and
\begin{equation}
    \begin{split}
        \Gamma w_{i} + \mathbf{a}^0_{i}\mathbf{x} + b^0_{i} &\leq 0 \\
        - b^l_{i} - \mathbf{a}^l_{i}\mathbf{x} &\leq w_{i},~\forall l \in 1,...,L\\
        b^l_{i} + \mathbf{a}^l_{i}\mathbf{x} &\leq w_{i},~\forall l \in 1,...,L\\
    \end{split}
    \label{rom_linear}
\end{equation}
if $\mathbb{A}^l \neq 0$. Note that for this type of uncertainty, the robust constraints are linear.

\section{Introduction}

Aircraft design exists in a niche of design problems where "failure is
not an option"\footnote{Quoting Gene Kranz, the mission director of Apollo 13.}.
This is remarkable since aircraft design problems are rife with uncertainty about
technological capabilities, environmental factors, manufacturing quality and the
state of markets and regulatory agencies. Since the program risk of aircraft design
problems is high, optimization under uncertainty for aircraft presents a lot of
low hanging fruit, since its goal is to be able to provide designs that are robust
to realizations of uncertainty in the real world.

Zang et al.\cite{Zang2002} succinctly describe the categories of benefits for optimization under uncertainty for aircraft.
These are the following:
\begin{itemize}
    \item \textit{Confidence in analysis tools will increase.}
    The uptake of new design tools in the aerospace industry has been low
    due to heavy reliance on legacy design methods and prior experience when
    faced with risky design propositions,
    even in the design of novel configurations where the understanding
    of the design tradespaces is lacking. Robustness will increase
    confidence in analysis tools because of its ability to better capture the
    effects of technological uncertainty on the potential benefits of new
    configurations.
    \item \textit{Design cycle time, cost, and risk will be reduced.}
    Design cycle costs as well as the engineering hours per aircraft have been increasing~\cite{Patt2012}.
    Aircraft design and development is costly, so the ability to handle uncertainty in
    the conceptual design process is critical for the long-term success of an aircraft,
    helping reduce the program risk.
    \item \textit{System performance will increase while ensuring that reliability requirements
    are met. }
    The effectiveness of an aircraft depends heavily on its
    ability to deliver on performance, which is dependent on assumptions about the
    current technological environment and the ability to produce vehicles of a certain quality.
    \item \textit{Designs will be more robust.}
    The ability to provide designs with feasibility and performance guarantees will mean
    that designs and products will be more robust to uncertainties in manufacturing quality,
    environmental factors, technology level and markets.
\end{itemize}

In economics, the idea that risk is related to profit is well understood and leveraged.
In aerospace engineering however, we often forget that there is no such thing as a free lunch,
and that the consequence of risk-aversity is often performance that is left on the table.
Considering that conceptual design in the aerospace industry hedges against program risk,
the \gls{ro} frameworks proposed in this paper will
give aerospace engineers the ability to rigorously trade robustness and the performance penalties
that result from it.

\subsection{Approaches to optimization under uncertainty}

Faced with the challenge of developing general nonlinear programs that can incorporate uncertainty,
the aerospace field has developed a number of mathematically non-rigorous methods to
design under uncertainty. Oftentimes, aerospace engineers will implement
\textit{margins} in the design process to account for uncertainties in parameters that a design's feasibility
may be sensitive to, such as material properties or maximum lift coefficient.
Another traditional method of adding robustness is through \textit{multi-mission design}~\cite{York2018},
which ensures that the aircraft is able to handle
multiple kinds of missions in the presence of no uncertainty. This is a type of \textit{finitely
adaptive} optimization geared to ensure objective performance in off-nominal operations.

The weaknesses of these non-rigorous methods are many. They provide no quantitative measures of
robustness or reliability~\cite{Zang2002}. Furthermore, they rely on the expertise of an experienced
engineer to guide the design process, without explicit knowledge of the trade-off between
robustness and optimality~\cite{Yao2011}. This is a dangerous proposition especially in the
conceptual design phase of new configurations, since prior information and expertise is not
available. In these scenarios, it is especially important to go back to fundamental physics
and use rigorous mathematics to explore the design space~\cite{York2018}.

There are two rigorous approaches to solving design optimization problems under uncertainty,
which are \gls{so} and \gls{ro}. Stochastic optimization\footnote{Note that stochastic
optimization is an overloaded term, and exists in two contexts in the literature. The first is the solution
of deterministic problems with stochastic search space exploration. The other is the solution
of problems of stochastic uncertainty. We explore the latter.}
deals with probability distributions of
uncertain parameters by propagating these uncertainties through the
physics of a design problem to ensure constraint feasibility with certain probabilistic guarantees.
The goal of \gls{so} is predominantly to minimize the expectation of an objective function, or
to optimize on some desired characteristic of the probability density function of the objective~\cite{Diwekar2008}.
Robust optimization takes a different approach, instead choosing to make designs immune to
uncertainties in parameters as long as the parameter values come from within the defined
uncertainty set. As such, \gls{ro} minimizes the worst-case objective outcome from a defined uncertainty set.

\subsection{Advantages of robust over stochastic optimization}

The formulation of stochastic models as \gls{ro} problems has many advantages
over general stochastic optimization methods, as explored in detail in~\cite{Bertsimas2011},
and fall into three categories, which are tractability, conservativeness and flexibility.
\gls{ro} is more tractable than \gls{so} due to the nature of uncertainty propagation.
General stochastic methods involve the propagation of uncertainties throughout a model
to determine their effects on constraint feasibility and the objective function.
This requires the integration of the product of probability distributions with potential outcomes,
and since the integration of continuous functions is difficult this is often achieved through
a combination of multi-dimensional quadrature and discretizations of the uncertainty into
possible scenarios. The propagation of parameter
scenarios results in a combinatorial explosion of possible outcomes which need to be evaluated to determine constraint
satisfaction and the distribution of the objective.

Few problems can be addressed purely through stochastic optimization (eg. the recourse problem as
shown in~\cite{Kall1982},\cite{Higle1991}, and energy planning problem such as in~\cite{Pereira1991}).
It is arguable that the methods used are still limited by the combinatorial
explosion of possible outcomes. \gls{ro} has to deal with a somewhat related problem, which is the issue of an infinite number
of possible realizations of constraints within a given constraint set. However, this is easily
tackled by considering the worst case robust counterpart of each constraint, which
results in many kinds of optimization problems having tractable robust formulations~\cite{Bertsimas2011}.

Although \gls{ro} problems solve problems with uncertainty,
\gls{ro} formulations result in solutions that are deterministically immune~\cite{Bertsimas2011}
to all possible realizations of parameters in an uncertainty set, which is defined as conservativeness.
\gls{so} formulations
provide no such guarantees. \gls{ro} also does not require distributional information
about uncertain parameters as \gls{so} does, and therefore can better address problems where there
is a lack of experience or data. It is arguable that \gls{ro}
leaves a lot on the table by not taking advantage of distributional information,
however there is a body of research on distributionally robust optimization~\cite{Bertsimas2013}
which seeks to leverage existing data.

There is significantly greater flexibility in the formulation of robust versus stochastic models
since the methods proposed are more general. It is important to highlight that,
although both \gls{ro} and \gls{so} seek to address the problem
of optimization under uncertainty, they solve fundamentally different problems. In an ideal world where
we have a problem that is tractable with global optimality for both methods, the two different
approaches would result in different solutions.

\subsection{Geometric and signomial programming for engineering design}

Geometric programming\footnote{Programming refers to the mathematical formulation of an optimization problem.}
is a method of log-convex optimization that has been developed
to solve problems in engineering design~\cite{Duffin1967}. Although theory of the \gls{gp} has existed since
the 1960's, \gls{gp}s have recently experienced a resurgence due to the advent of polynomial-time
interior point methods~\cite{Nesterov1994} and improvements in computing. They have been
applied to a range of engineering design problems with success. For a non-exhaustive list of examples,
please refer to~\cite{Boyd2007}.

\gls{gp}s have been effective in aircraft conceptual design
(\cite{Hoburg2013},~\cite{Burton2017}).
However, the stringent mathematical requirements of a \gls{gp} limits its application to non-log-convex problems.
The \gls{sp} is the difference-of-log-convex extension of the \gls{gp} which can be applied to
solve this larger set of problems, albeit with the loss of some mathematical guarantees compared to the \gls{gp}~\cite{Kirschen2018}.
Aircraft pose some of the most challenging design problems~\cite{York2018}, and signomial programming
has been used to great effect in modeling and designing complex aircraft at a conceptual level quickly
and reliably as in \cite{York2018}, \cite{Kirschen2016} and \cite{Kirschen2018}.
Other interesting applications for SPs such as in network flow problems are being investigated.

Robust formulations exist for solving geometric programs with parametric uncertainty~\cite{Saab2018}.
We posit that the creation of a robust signomial programming framework to capture uncertainty in engineering
design, and specifically aircraft design, will allow us to have more confidence in the results
of the conceptual design phase, reduce program risk, and increase overall system performance.

\subsection{Contributions}

In this paper, we propose a tractable \gls{rsp} which we solve as a sequential \gls{rgp},
allowing us to implement robustness in non-log-convex problems such as aircraft design.
We extend the \gls{rgp} framework developed by Saab~\cite{Saab2018} to \gls{sp}s.
We implement the \gls{rsp} formulation on a simple aircraft design problem with several hundred
variables as defined in~\cite{Ozturk2018}.
We demonstrate the benefits of robust optimization both in ensuring design feasibility and performance
in off-nominal conditions. We further explore the benefits of \gls{ro} in multiobjective optimization.




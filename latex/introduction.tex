\section{Introduction}

\textbf{Why optimization under uncertainty for aircraft}


\cite{Zang2002} succinctly described the categories of benefits for optimization under uncertainty for aircraft.
The following summarizes and qualitatively explores the benefits.
\begin{itemize}
    \item \textit{Confidence in analysis tools will increase.}
    \item \textit{Design cycle time, cost, and risk will be reduced.}
    Design cycle costs of aerospace vehicles have been increasing. This has to do
    many factors, such as the growth of requirements (add here and cite)...
    Aircraft design and development is costly, so the ability to handle uncertainty in
    the conceptual design process is critical for the long-term success of an aircraft,
    helping reduce the program risk.
    \item \textit{System performance will increase while ensuring that reliability requirements
                  are met. }
    The effectiveness of an aircraft depends heavily on its
    ability to deliver on performance, which is dependent on assumptions about the
    current technological environment and the ability to produce vehicles of a certain quality.
    \item \textit{Designs will be more robust.}
    The ability to provide designs with feasibility and performance guarantees...
\end{itemize}

In economics, the idea that risk is related to profit is well understood and leveraged.
In aerospace engineering however, we often forget that there is no such thing as a free lunch,
and that the consequence of risk-aversity is often performance that is left on the table.
Good conceptual design in the aerospace industry hedges against program risk,
the \gls{ro} frameworks proposed in this paper will
give aerospace engineers the ability to rigorously trade robustness and the performance penalties
that result from it.

\subsection{Approaches to optimization under uncertainty}

Faced with the challenge of developing general \gls{nlp}s that can incorporate uncertainty,
the aerospace field has developed a number of mathematically non-rigorous methods to
design under uncertainty. Oftentimes, aerospace engineers will implement
\textit{margins} in the design process to account for uncertainties in parameters that a design's feasibility
may be sensitive to, such as material properties or maximum lift coefficient.
Another traditional method of adding robustness is through doing \textit{multi-mission design}
(TODO: cite York et. al here), which ensures that the aircraft is able to handle
multiple kinds of missions in the presence of no uncertainty. This is a type of \textit{finitely
adaptive} optimization geared to ensure objective performance in off-nominal operations.

The weaknesses of these non-rigorous methods are many. They provide no quantitative measures of
robustness or reliability~\cite{Zang2002}. Furthermore, they rely on the expertise of an experienced
engineer to guide the design process, without explicit knowledge of the trade-off between
robustness and optimality~\cite{yao2011}. This is a dangerous proposition especially in the
conceptual design phase of new configurations, since prior information and expertise is not
available. In these scenarios, it is especially important to go back to fundamental physics
and use rigorous mathematics to explore the design space. (cite York here)

There are two rigorous approaches to solving design optimization problems under uncertainty,
which are \gls{so} and \gls{ro}. Stochastic optimization\footnote{Note that stochastic
optimization is an overloaded term, and exists in two contexts in the literature. The first is the solution
of deterministic problems with stochastic search space exploration. The other is the solution
of problems of stochastic uncertainty. We explore the latter.}
deals with probability distributions of
uncertain parameters by propagating these uncertainties through the
physics of a design problem to ensure constraint feasibility with certain probabilistic guarantees.
The goal of \gls{so} is minimize the expectation of an objective function~\cite{Diwekar2008}.
Robust optimization takes a different approach, instead choosing to make designs immune to
uncertainties in parameters as long as the parameter values come from within the defined
uncertainty set. \gls{ro} minimizes the worst-case objective outcome from a defined uncertainty set.

\subsection{Advantages of robust over stochastic optimization}

The formulation of stochastic models as \gls{ro} problems has many advantages
over general stochastic optimization methods, as summarized in~\cite{Bertsimas2011},
and fall into three categories, which are tractability, conservativeness and flexibility.
\gls{ro} is more tractable than \gls{so} due to the nature of uncertainty propagation.
General stochastic methods involve the propagation of uncertainties throughout a model
to determine their effects on constraint feasibility and the objective function.
This requires the integration of the product of probability distributions with potential outcomes,
and since the integration of continuous functions is difficult this is often achieved through
a discretization of the uncertainty into possible scenarios. The propagation of parameter
scenarios results in a combinatorial explosion of possible outcomes which need to be evaluated to determine constraint
satisfaction and the distribution of the objective.

Few problems can be addressed purely through stochastic optimization (eg. the recourse problem as
shown in~\cite{Kall1982},\cite{Higle1991}, and energy planning problem such as in~\cite{Pereira1991}).
It is arguable that the methods used are still limited by the combinatorial
explosion of possible outcomes. \gls{ro} has to deal with a somewhat related problem, which is the issue of an infinite number
of possible realizations of constraints within a given constraint set. However, this is easily
tackled by considering the worst case robust counterpart of each constraint, which
results in many kinds of optimization problems having tractable robust formulations~\cite{Bertsimas2011}.

Although \gls{ro} problems solve problems with uncertainty,
\gls{ro} formulations result in solutions that are deterministically immune~\cite{Bertsimas2011}
to all possible realizations of parameters in an uncertainty set, which is defined as conservativeness.
\gls{so} formulations
provide no such guarantees. \gls{ro} also does not require distributional information
about uncertain parameters as \gls{so} does, and therefore can better address problems where there
is a lack of experience or data. It is arguable that \gls{ro}
leaves a lot on the table by not taking advantage of distributional information,
however there is a body of research on distributionally robust optimization~\cite{Bertsimas2013}
which seeks to leverage existing data.

There is significantly greater flexibility in the formulation of robust versus stochastic models
since the methods proposed are more general. It is important to highlight that,
although both \gls{ro} and \gls{so} seek to address the problem
of optimization under uncertainty, they solve fundamentally different problems. In an ideal world where
we have a problem that is tractable with global optimality for both methods, the two different
approaches would result in different solutions.

\subsection{Geometric and signomial programming for engineering design}

Geometric programming is a method of log-convex optimization for which robust formulations exist~\cite{Saab2018}.
However, the stringent mathematical requirements of a \gls{gp} limits its application to non-log-convex problems.
The \gls{sp} is the difference-of-log-convex extension of the \gls{gp} which can be applied to
solve this larger set of problems, albeit with the loss of some mathematical guarantees compared to the \gls{gp}~\cite{Kirschen2017}.
In this paper, we propose a tractable \gls{rsp} which we solve as a sequential \gls{rgp},
allowing us to implement robustness in non-log-convex problems.

\textbf{Why are SPs good for aircraft design?}

Aircraft pose some of the most challenging design problems. They have physics that span many disciplines,
and have many often conflicting objectives.

\gls{gp}s have been demonstrated to be effective in aircraft conceptual design
(\cite{Hoburg2013},~\cite{Burton2017},~\cite{Kirschen2017},CITE York Ozturk here).

\textbf{Why is robust optimization a more rigorous method to integrate uncertainty in aircraft optimization?}



\subsection{Contributions}

We extend the \gls{rgp} framework developed by Saab~\cite{Saab2018} to \gls{sp}s.
We implement the \gls{rsp} formulation on a simple aircraft design problem with 19 free variables,
12 uncertain parameters and 17 constraints to demonstrate its potential (TODO: update).
We demonstrate the benefits of robust optimization both in ensuring design feasibility and performance
in off-nominal conditions. We further explore the benefits of \gls{ro} in multiobjective optimization.

\section{Mathematical Background}
\subsection{Geometric Programming}
A \textbf{geometric program in posynomial form} is a log-convex optimization problem of the form:
\begin{equation}
\begin{aligned}
	& \text{minimize} && f_0 \left(\vec{u}\right) \\
	& \text{subject to} && f_i \left(\vec{u}\right) \leq 1, i = 1,...,m_p\\
	& && h_i \left(\vec{u}\right) = 1, i = 1, ...,m_e\\
\end{aligned}
\label{GP_standard}
\end{equation}
where each $f_i$ is a {\em posynomial}, each $h_i$ is a {\em monomial}, $m_p$ is the number of posynomials, and $m_e$ is the number of monomials. A monomial $h(\vec{u})$ is a function of the form:
\begin{displaymath}
	h_i(\vec{u}) = e^{b_i}\textstyle{\prod}_{j=1}^{n}{u_j}^{a_{ij}}
\end{displaymath}
where $a_{ij}$ is the $j^{th}$ component of a row vector $\vec{a_i}$ in $\mathbb{R}^n$, $u_j$ is the $j^{th}$ component of a column vector $\vec{u}$ in $\mathbb{R}^n_+$ , and $b_i$ is in $\mathbb{R}$. A posynomial $f(\vec{u})$ is the sum of $K \in \mathbb{Z}^+$ monomials:
\begin{displaymath}
	f_i(\vec{u}) = \textstyle{\sum_{k=1}^{K}}e^{b_{ikj}}\prod_{j=1}^{n}{u_j}^{a_{ikj}}
\end{displaymath}
where $a_{ikj}$ is the $j^{th}$ component of a row vector $\vec{a_{ik}}$ in $\mathbb{R}^n$, $u_j$ is the $j^{th}$ component of a column vector $\vec{u}$ in $\mathbb{R}^n_+$, and $b_{ik}$ is in $\mathbb{R}$ \cite{GP_tutorial}.\\
A logarithmic change of the variables $x_j = \log(u_j)$ would turn a monomial into {\em  the exponential of an affine function} and a posynomial into {\em the sum of exponentials of affine functions}. A transformed monomial $h_i(\vec{x})$ is a function of the form:
\begin{displaymath}
    h_i(\vec{x}) = e^{\vec{a_i}\vec{x} + b_i}
\end{displaymath}
where $\vec{x}$ is a column vector in $\mathbb{R}^n$. A transformed posynomial $f_i(\vec{x})$ is the sum of $K_i \in \mathbb{Z}^+$ monomials:
\begin{displaymath}
    f_i(\vec{x}) = \textstyle{\sum_{k=1}^{K_i}}e^{\vec{a_{ik}}\vec{x} + b_{ik}}
\end{displaymath}
where $\vec{x}$ is a column vector in $\mathbb{R}^n$. A geometric program with transformed constraints is a \textbf{geometric program in exponential form}.

The positive nature of exponential functions restricts the space spanned by posynomials and limits the applications of \gls{gp}s to certain classes of problems. The limited applicability of \gls{gp}s has motivated the introduction of signomials.

\subsection{Signomial Programming}
A {\em signomial} can be defined as the difference between two posynomials, consequently, an SP is a non-log-convex optimization problem of the form:
\begin{equation}
\begin{aligned}
&\text{minimize } && f_{0}(\vec{x}) \\
&\text{subject to } && f_{i}(\vec{x}) - g_{i}(\vec{x})& \leq 0, i = 1, ...., m \\
\end{aligned}
\end{equation} 
where $f_{i}$ and $g_{i}$ are both posynomials, and $\vec{x}$ is a column vector in $\mathbb{R}^n$. 

Reliably solving an SP to a local optimum has been described in \cite{GP_tutorial} and \cite{lipp_boyd_2015}. A common solution heuristic involves solving an SP as a sequence of GPs, where each GP is a local approximation of the SP. Although it is a powerful tool, applications involving SPs are usually prone to uncertainties that have a significant effect on the solution.

\subsection{Overview}

TODO: General optimization techniques for aircraft design

Signomial programming can cover constraints that might be neither linear, convex, nor log-covex and, hence, it can be used to model problems that cannot be formulated by standard optimization tools such as linear or geometric programs. Although global optimal solutions are not guaranteed, however, signomial programming is a powerful tool that is currently being used in modeling and solving complex aircraft designs quickly and reliably as in \cite{york_hoburg_drela_2018} \cite{kirschen_burnell_hoburg_2016} \cite{kirschen_york_ozturk_hoburg_2018}. Other interesting applications for SPs such as in network flow problems are being investigated.

TODO: Motivate the use of robust opt. for aircraft design over traditional methods, and stochastic/UQ.

TODO: More depth/references as to methods for UQ/RO in aircraft design.

TODO: Sections and outline

\section{Introduction}

\textbf{Why optimization under uncertainty for aircraft}
\newline
\textbf{Approaches to optimization under uncertainty}

There are two major approaches to solving design optimization problems under uncertainty,
which are \gls{so} and \gls{ro}. Stochastic optimization deals with probability distributions of
uncertain parameters by propagating these uncertainties through the
physics of a design problem to ensure constraint feasibility with certain probabilistic guarantees.
The goal of \gls{so} is minimize the expectation of an objective function~\cite{Diwekar2008}.
Robust optimization takes a different approach, instead choosing to make designs immune to
uncertainties in parameters as long as the parameter values come from within the defined
uncertainty set. \gls{ro} minimizes the worst-case objective outcome from a defined uncertainty set.

\subsection{Advantages of robust over stochastic optimization}

The formulation of stochastic models as \gls{ro} problems has many advantages
over general stochastic optimization methods, as summarized in~\cite{Bertsimas2011},
and fall into three categories, which are tractability, conservativeness and flexibility.

\gls{ro} is more tractable than \gls{so} due to the nature of uncertainty propagation.
General stochastic methods involve the propagation of uncertainties throughout the design
to determine their effects on constraint feasibility and the objective function.
This is easier said than done.
This requires the integration of probability distributions,
and since the integration of continuous functions is difficult often achieved through
a discretization of the uncertainty into possible scenarios.


The solution of the \gls{ro} problem is immune to the variations of the uncertain
parameters defined in the uncertainty set.

Many kinds of optimization problems have tractable robust formulations, while tractable stochastic
optimization is few and far between. \gls{so} requires the
distributional information of the parameters of the problem to solve, while \gls{ro} circumvents
this issue altogether by defining uncertainty bounds. There are few problems that can
be addressed through purely stochastic optimization (eg. the recourse problem as
shown in~\cite{Kall1982},\cite{Higle1991}).

An important note is that, although both \gls{ro} and \gls{so} seek to address the problem
of optimization under uncertainty, they solve different problems. In an ideal world where
we have a problem that is tractable with global optimality for both methods, the solutions
of the two would not match.

They are different approaches as well as

\subsection{Geometric and signomial programming for engineering design}

Geometric programming is a method of log-convex optimization for which robust formulations exist
(TODO: cite Saab here).
However, the stringent mathematical requirements of a \gls{gp} limits its application to non-log-convex problems.
The \gls{sp} is the difference-of-log-convex extension of the \gls{gp} which can be applied to
solve this larger set of problems, albeit with the loss of some mathematical guarantees compared to the \gls{gp}.
In this paper, we propose a tractable \gls{rsp} which we solve as a sequential \gls{rgp},
allowing us to implement robustness in non-log-convex problems.

\textbf{Why are SPs good for aircraft design?}

Aircraft pose some of the most challenging design problems. They have physics that span many disciplines,
and have many often conflicting objectives.

\textbf{Why is robust optimization a more rigorous method to integrate uncertainty in aircraft optimization?}

The effectiveness of an aircraft depends heavily on its
ability to deliver on performance, which is heavily dependent on assumptions about the
current technological environment and the ability to produce vehicles of a certain quality
Furthermore, aircraft design and development is costly, so the ability to handle uncertainty in
the conceptual design process is critical for the long-term success of an aircraft program.

TODO: Motivate the use of robust opt. for aircraft design over traditional methods, and stochastic/UQ.

Oftentimes, aerospace engineers will implement
margins in the design process to account for uncertainties in parameters that a design may be sensitive to,
without explicit knowledge of the trade-off between robustness and optimality~\cite{yao2011}.

Another method of adding robustness to aircraft design is through the use of multi-mission
optimization (TODO: cite York et. al here), being that the aircraft is guaranteed to be able to handle
multiple kinds of missions in the presence of no uncertainty. This is a type of \textit{adaptively
robust} optimization geared towards

All of these methods rely on the expertise of an experienced engineer to guide the design process.

In economics, the idea that risk is related to profit is well understood and leveraged.
In aerospace engineering however, we often forget that there is no such thing as a free lunch,
and that the consequence of risk-aversity
is often performance that is left on the table. The \gls{ro} frameworks proposed in this paper will
give aerospace engineers the ability to rigorously trade robustness and the performance penalties
that result from it.

\subsection{Contributions}

We extend the \gls{rgp} framework developed by Saab (TODO: cite) to \gls{sp}s.

We implement the \gls{rsp} formulation on a simple aircraft design problem with 19 free variables,
12 uncertain parameters and 17 constraints to demonstrate its potential (TODO: update).

A robust aircraft design formulation will allow designers to allocate margin more effectively
to obtain better-performing designs with feasibility guarantees.


The classical method of accounting for uncertainty in aerospace engineering is through
the use of \textit{margins}, which are essentially

This requires a skilled engineer to

However, this begs the question, who decides on the margins and

Robust optimization also relies on an engineer's intuition about the underlying uncertainty,

\section{Mathematical Background}
\subsection{Geometric Programming}
A \textbf{geometric program in posynomial form} is a log-convex optimization problem of the form:
\begin{equation}
\begin{aligned}
	& \text{minimize} && f_0 \left(\vec{u}\right) \\
	& \text{subject to} && f_i \left(\vec{u}\right) \leq 1, i = 1,...,m_p\\
	& && h_i \left(\vec{u}\right) = 1, i = 1, ...,m_e\\
\end{aligned}
\label{GP_standard}
\end{equation}
where each $f_i$ is a {\em posynomial}, each $h_i$ is a {\em monomial}, $m_p$ is the number of posynomials, and $m_e$ is the number of monomials. A monomial $h(\vec{u})$ is a function of the form:
\begin{displaymath}
	h_i(\vec{u}) = e^{b_i}\textstyle{\prod}_{j=1}^{n}{u_j}^{a_{ij}}
\end{displaymath}
where $a_{ij}$ is the $j^{th}$ component of a row vector $\vec{a_i}$ in $\mathbb{R}^n$, $u_j$ is the $j^{th}$ component of a column vector $\vec{u}$ in $\mathbb{R}^n_+$ , and $b_i$ is in $\mathbb{R}$. A posynomial $f(\vec{u})$ is the sum of $K \in \mathbb{Z}^+$ monomials:
\begin{displaymath}
	f_i(\vec{u}) = \textstyle{\sum_{k=1}^{K}}e^{b_{ikj}}\prod_{j=1}^{n}{u_j}^{a_{ikj}}
\end{displaymath}
where $a_{ikj}$ is the $j^{th}$ component of a row vector $\vec{a_{ik}}$ in $\mathbb{R}^n$, $u_j$ is the $j^{th}$ component of a column vector $\vec{u}$ in $\mathbb{R}^n_+$, and $b_{ik}$ is in $\mathbb{R}$ \cite{GP_tutorial}.\\
A logarithmic change of the variables $x_j = \log(u_j)$ would turn a monomial into {\em  the exponential of an affine function} and a posynomial into {\em the sum of exponentials of affine functions}. A transformed monomial $h_i(\vec{x})$ is a function of the form:
\begin{displaymath}
    h_i(\vec{x}) = e^{\vec{a_i}\vec{x} + b_i}
\end{displaymath}
where $\vec{x}$ is a column vector in $\mathbb{R}^n$. A transformed posynomial $f_i(\vec{x})$ is the sum of $K_i \in \mathbb{Z}^+$ monomials:
\begin{displaymath}
    f_i(\vec{x}) = \textstyle{\sum_{k=1}^{K_i}}e^{\vec{a_{ik}}\vec{x} + b_{ik}}
\end{displaymath}
where $\vec{x}$ is a column vector in $\mathbb{R}^n$. A geometric program with transformed constraints is a \textbf{geometric program in exponential form}.

The positive nature of exponential functions restricts the space spanned by posynomials and limits the applications of \gls{gp}s to certain classes of problems. The limited applicability of \gls{gp}s has motivated the introduction of signomials.

\subsection{Signomial Programming}
A {\em signomial} can be defined as the difference between two posynomials, consequently, an SP is a non-log-convex optimization problem of the form:
\begin{equation}
\begin{aligned}
&\text{minimize } && f_{0}(\vec{x}) \\
&\text{subject to } && f_{i}(\vec{x}) - g_{i}(\vec{x})& \leq 0, i = 1, ...., m \\
\end{aligned}
\end{equation} 
where $f_{i}$ and $g_{i}$ are both posynomials, and $\vec{x}$ is a column vector in $\mathbb{R}^n$. 

Reliably solving an SP to a local optimum has been described in \cite{GP_tutorial} and \cite{lipp_boyd_2015}. A common solution heuristic involves solving an SP as a sequence of GPs, where each GP is a local approximation of the SP. Although it is a powerful tool, applications involving SPs are usually prone to uncertainties that have a significant effect on the solution.

\subsection{Overview}

TODO: General optimization techniques for aircraft design

Signomial programming can cover constraints that might be neither linear, convex, nor log-covex and, hence, it can be used to model problems that cannot be formulated by standard optimization tools such as linear or geometric programs. Although global optimal solutions are not guaranteed, however, signomial programming is a powerful tool that is currently being used in modeling and solving complex aircraft designs quickly and reliably as in \cite{york_hoburg_drela_2018} \cite{kirschen_burnell_hoburg_2016} \cite{kirschen_york_ozturk_hoburg_2018}. Other interesting applications for SPs such as in network flow problems are being investigated.

TODO: Motivate the use of robust opt. for aircraft design over traditional methods, and stochastic/UQ.

TODO: More depth/references as to methods for UQ/RO in aircraft design.

TODO: Sections and outline

\section{Introduction}

Robust optimization methods provide tractable methods to capture uncertainty in design.

The advantages of robust optimization methods over stochastic optimization methods
for optimization under uncertainty are summarized in \cite{Bertsimas2011}, 
and 

TODO: Motivate the use of robust opt. for aircraft design over traditional methods, and stochastic/UQ.

Geometric programming is a method of log-convex optimization for which robust formulations exist.
However, the stringent mathematical requirements of a \gls{gp} limits its application to non-log-convex problems.
The \gls{sp} is the difference-of-log-convex extension of the \gls{gp} which can be applied to
solve this larger set of problems, albeit with the loss of some mathematical guarantees compared to the \gls{gp}.
In this paper, we propose a tractable \gls{rsp} which we solve as a sequential \gls{rgp},
allowing us to implement robustness in non-log-convex problems.

We implement the \gls{rsp} formulation on a simple aircraft design problem with 19 free variables,
12 uncertain parameters and 17 constraints to demonstrate its potential. We believe that aircraft design
problems can especially benefit from robustness. Oftentimes, aerospace engineers will implement
margins in the design process to account for uncertainties in parameters that a design may be sensitive to,
without explicit knowledge of the trade-off between robustness and optimality~\cite{yao2011}.

TODO: More depth/references as to methods for UQ/RO in aircraft design.

A robust aircraft design formulation will allow designers to allocate margin more effectively
to obtain better-performing designs with feasibility guarantees.

\subsection{Geometric Programming}

In its \textbf{natural form}, a \gls{gp} is a log-convex optimization problem of the form
\begin{equation}
\begin{aligned}
	& \text{minimize} && f_0 \left(\vec{u}\right) \\
	& \text{subject to} && f_i \left(\vec{u}\right) \leq 1, i = 1,...,m_p\\
	& && h_i \left(\vec{u}\right) = 1, i = 1, ...,m_e\\
\end{aligned}
\label{GP_standard}
\end{equation}
where each $f_i$ is a {\em posynomial} and each $h_i$ is a {\em monomial}. A monomial $h(\vec{u})$ is a function of the form
\begin{displaymath}
	h(\vec{u}) = e^{b}\textstyle{\prod}_{j=1}^{n}{u_j}^{a_j}
\end{displaymath}
where $\vec{a}$ is a row vector in $\mathbb{R}^n$, $\vec{u}$ is a column vector in $\mathbb{R}^n_+$ , and $b$ is in $\mathbb{R}$. A posynomial $f(\vec{u})$ is the sum of $K \in \mathbb{Z}^+$ monomials
\begin{displaymath}
	f(\vec{u}) = \textstyle{\sum_{k=1}^{K}}e^{b_{kj}}\prod_{j=1}^{n}{u_j}^{a_{kj}}
\end{displaymath}
where $\vec{a_{k}}$ is a row vector in $\mathbb{R}^n$, $\vec{u}$ is a column vector in $\mathbb{R}^n_+$, and $b_{k}$ is in $\mathbb{R}$ \cite{GP_tutorial}.\\
A logarithmic change of the variables $x_j = \log(u_j)$ would convert the \gls{gp} into its \textbf{exponential form}. A monomial $h_i(\vec{x})$ would then become a function of the form
\begin{displaymath}
    h_i(\vec{x}) = e^{\vec{a_i}\vec{x} + b_i}
\end{displaymath}
where $\vec{a_i}$ is a row vector in $\mathbb{R}^n$, $\vec{x}$ is a column vector in $\mathbb{R}^n$ , and $b_i$ is in $\mathbb{R}$. While a posynomial $f_i(\vec{x})$ would become a function of the form
\begin{displaymath}
    f_i(\vec{x}) = \textstyle{\sum_{k=1}^{K_i}}e^{\vec{a_{ik}}\vec{x} + b_{ik}}
\end{displaymath}
where $\vec{a_{ik}}$ is a row vector in $\mathbb{R}^n$, $\vec{x}$ is a column vector in $\mathbb{R}^n$, and $b_{ik}$ is in $\mathbb{R}$.

Any monomial constraint $h(\vec{x}) = 1$ can be replaced by $h(\vec{x}) \leq 1$ and $\frac{1}{h(\vec{x})} \leq 1$. Indeed, this replacement leads to the following \textbf{inequality form} of a \gls{gp}

\begin{equation}
\begin{aligned}
& \text{minimize} && \textstyle{\sum}_{k=1}^{K_0}e^{\vec{a}_{0k}\vec{x} + b_{0k}} \\
& \text{subject to} && \textstyle{\sum}_{k=1}^{K_i}e^{\vec{a}_{ik}\vec{x} + b_{ik}} \leq 1 \quad \forall i \in 1,...,m\\
\end{aligned}
\label{GP_inequality}
\end{equation}

% A tractable approximation of the \gls{rgp} exists and will be discussed, and has been formulated with polyhedral and ellipsoidal uncertainty sets. \gls{rgp}s allow for uncertainty in both the coefficients $b_k$ and the exponents $\mathbf{a}_{k}$. However, t

The positive nature of exponential functions restricts the space spanned by posynomials and limits the applications of \gls{gp}s to certain classes of problems. The limited applicability of \gls{gp}s has motivated the introduction of signomials.
% To be able to find robust solutions to \gls{sp}s, we would like to create a framework to generalize the
% uncertainty sets used in \gls{rgp}s to \gls{rsp}s.
\subsection{Signomial Programming}

A {\em signomial} can be defined as the difference between two posynomials, consequently, an \gls{sp} is a non-log-convex optimization problem of the form
\begin{equation}
\begin{aligned}
&\text{minimize } && f_{0}(\vec{x}) \\
&\text{subject to } && f_{i}(\vec{x}) - g_{i}(\vec{x})& \leq 0, i = 1, ...., m \\
\end{aligned}
\end{equation} 
where $f_{i}$ and $g_{i}$ are both posynomials. This problem can be solved as sequential geometric programs, i.e. by taking the monomial approximation of $g_{i}$ at a solution or initial guess $x_{i}$, solving the resulting \gls{gp} approximation to obtain a new solution $x_{i+1}$, and repeating the process until some convergence parameter is satisfied.

The SP algorithm is well-studied, reliably solving \gls{sp}s with an initial guess of all 1’s. Moreover, \gls{sp}s are guaranteed to be sub-optimal but feasible solutions to a general non-linear problem.
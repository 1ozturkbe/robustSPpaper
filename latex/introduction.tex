\section{Introduction}

Aircraft design exists in a niche of design problems where ``failure is
not an option"\footnote{Quoting Gene Kranz, the mission director of Apollo 13.}.
This is remarkable since aircraft design problems are rife with uncertainty about
technological capabilities, environmental factors, manufacturing quality and the future
state of markets and regulatory agencies.
Optimization under uncertainty seeks to provide designs that are robust
to realizations of uncertainty in the real world and can reduce
the high risk of aerospace programs.

Optimization has become ubiquitous in the design of engineered systems, and especially aerospace systems,
in the late 20th and 21st centuries as computing has improved dramatically and as designs have
continued to approach the limits of the second law of thermodynamics. Optimization under uncertainty
has been identified by academia and industry as an area of opportunity
in multiple review papers (\cite{Zang2002},~\cite{Yao2011}),
and we elaborate on four potential benefits from \cite{Zang2002} below:
\begin{itemize}
    \item \emph{Confidence in analysis tools will increase.}
    The uptake of new design tools in the aerospace industry has been low
    due to heavy reliance on legacy design methods and prior experience when
    faced with risky design propositions, and notably in
    the design of novel configurations where understanding
    of the design tradespaces is lacking. Robustness will increase
    confidence in analysis tools because it appropriately captures the
    effects of technological uncertainty on the potential benefits of new
    configurations.
    \item \emph{Design cycle time, cost, and risk will be reduced.}
    Design cycle costs as well as the engineering hours per aircraft have been increasing
    precipitously since the 1950's~\cite{Patt2012}.
    Aircraft design and development is costly, so the ability to handle uncertainty in
    the conceptual design process is critical for the long-term success of an aircraft,
    helping reduce the program risk.
    \item \emph{Designs will be more robust.}
    The ability to provide designs with feasibility guarantees will mean
    that designs and products will be more robust to uncertainties in manufacturing quality,
    environmental factors, technology level and markets, and better able to
    handle off-nominal operating conditions.
    \item \emph{System performance will increase while ensuring that reliability requirements
    are met.}
    Design under uncertainty will allow for a better understanding of the trade-off between risk and
    performance. As a result, it will allow for designs that are less conservative than
    traditional designs while meeting the same reliability requirements.
\end{itemize}

In economics, the idea that risk is related to profit is well understood and leveraged.
In aerospace engineering however we often forget that risk aversity necessarily results in lower performance.
Considering that conceptual design in the aerospace industry hedges against program risk,
the tractable \gls{ro} frameworks proposed in this paper will
give aerospace engineers the ability to rigorously trade-off robustness to uncertainty with the performance penalties
that result.

\subsection{Approaches to optimization under uncertainty}
\label{sec:approaches}

Faced with the challenge of finding designs that can handle uncertainty,
the aerospace field has developed a number of methods to
design under uncertainty. Oftentimes, aerospace engineers will implement
\emph{margins} in the design process to account for uncertainties in parameters that a design's feasibility
may be sensitive to, such as material properties or maximum lift coefficient.
Another traditional method of adding robustness is through \emph{multi-mission design}~\cite{York2018},
which ensures that the aircraft is able to handle
multiple kinds of missions in the presence of no uncertainty. This is a type of \emph{finitely
adaptive} optimization geared to ensure objective performance in off-nominal operations.

These methods have several weaknesses. They provide no quantitative measures of
robustness or reliability~\cite{Zang2002}. They rely on the expertise of an experienced
engineer to guide the design process, without explicit knowledge of the trade-off between
robustness and optimality~\cite{Yao2011}. This is a dangerous proposition especially in the
conceptual design phase of new configurations, since prior information and expertise is not
available. In these scenarios, it is especially important to go back to fundamental physics
and use rigorous mathematics to explore the design space~\cite{York2018}. Furthermore,
the legacy methods are often too conservative, ruling out potentially beneficial technologies
and configurations due to the inability to adequately trade off performance and risk.

There are two rigorous approaches to solving design optimization problems under uncertainty,
which are \gls{so} and \gls{ro}. Note that stochastic
optimization is an overloaded term, and exists in at least two contexts in the literature. The first is the solution
of deterministic problems with stochastic search space exploration. The second is the solution
of design optimization problems with stochastic parameters, which is the focus of this paper.
In this context, \gls{so} problems deal with probability distributions of
uncertain parameters by propagating them through the
physics of a design problem to ensure constraint feasibility with certain probabilistic guarantees.
The predominant goal of \gls{so} is to minimize some characteristics, for example moments or risk measures,
of the probability density function of the quantity of interest~\cite{Diwekar2008}.
In contrast, \gls{ro} takes a different approach, instead choosing to make designs immune to
uncertainties in parameters as long as the parameter values come from within a defined
uncertainty set. As such, \gls{ro} avoids the need to propagate entire probability
distributions by minimizing the worst-case objective outcome of a design for a
given set over the uncertain parameters.

\subsection{Comparison of robust and stochastic optimization methods for conceptual design}
\label{sec:robustvsstochastic}

Both \gls{ro} and \gls{so} have relative advantages in implementation. This paper will
argue specifically that the formulation of conceptual engineering design problems under uncertainty as
\gls{ro} problems has advantages over \gls{so} formulations (a more
mathematical programming centric comparison is made in~\cite{Bertsimas2011}).

\subsubsection{Generality and tractability}

In the context of engineering, we claim that an optimization method is general
when it can be used to solve a range of problems of interest. On the other hand,
tractability describes whether or not the problems are solved to a satisfactory
optimum with reasonable computational time. Optimization
under uncertainty is a difficult task that puts these two desirable subjective traits
at odds with each other.

\gls{so} has the advantage of generality.
\gls{so} methods are easily applicable to black box models or input-output systems.
They require little knowledge, if any, about the constraints in the system of interest.
\gls{ro} methods are less general, since they require
the design objective and constraints to be explicit and cast in a form that has a worst-case
counterpart. Thus models for \gls{ro} have to be transparent,
and \gls{ro} cannot be applied to black box models without significant prior data
manipulation at a minimum. A mitigating factor is that
many classes of conceptual engineering design problems can be cast or approximated in a form that
is compatible with robust optimization, such as linear, quadratic, semidefinite
and geometric programs.

On the other hand, \gls{ro} is more tractable than \gls{so} due to the difference in method of uncertainty propagation.
As aforementioned, stochastic methods involve the propagation of probability densities throughout a model
to determine their effects on constraint feasibility and the objective function.
This requires the integration of the product of probability distributions with potential outcomes,
and since the integration of continuous functions is difficult this is often achieved through
a combination of high-dimensional quadrature and discretizations of the uncertainty into
possible scenarios. The propagation of parameter
scenarios results in a combinatorial explosion of possible outcomes which need to be evaluated to determine constraint
satisfaction and the distribution of the objective. Few problems can be addressed purely
through stochastic optimization (eg. the recourse problem as
shown in~\cite{Kall1982},\cite{Higle1991}, and energy planning problem such as in~\cite{Pereira1991}), and
even these are limited by combinatorics and costly system evaluations. Furthermore, they require
problem-specific approximations, so that generality is compromised.
Robust versions of tractable optimization problem are not
guaranteed to be tractable, but in practice the aforementioned classes of optimization problems
have tractable robust formulations~\cite{Bertsimas2011}. In \gls{ro},
there are no separate optimization and evaluation
loops by construction, and thus \gls{ro} problems can be solved optimally
many orders of magnitude faster than \gls{so} problems of the same form~\cite{Bertsimas2011}.

Conceptual design optimization values generality, because engineers would like to
apply methods for optimization under uncertainty without significant mathematical groundwork,
and tractability, because fast solution times are critical
to reduce program risk early on in the design process when more aspects
of the design are fluid. From this perspective, the relative intractability of
\gls{so}-based approaches makes them unreliable for conceptual design, since significant time is
needed both to develop problem-specific tractable formulations, and to find satisfactory optima.
Furthermore,
many engineering design problems such as aircraft design are approximable by optimization
forms that have tractable robust counterparts, making \gls{ro} better suited
to conceptual design.

\subsubsection{Use of data}

\gls{so} problems generally require complete knowledge of the probability distribution of
parameters. During conceptual design, this data is not available or is unreliable.
\gls{ro} requires only `modest assumptions  about distributions, such as a known mean and
bounded support'~\cite{Chen2007}. Since \gls{ro} does not require as much information
about uncertain parameters as \gls{so} does, it can better address problems where there
is a lack of experience or data. It is arguable that \gls{ro}
leaves a lot on the table by not taking advantage of distributional information,
however there is a growing body of research on distributionally robust optimization~\cite{Bertsimas2017}
which seeks to leverage existing data.

During conceptual design, data about the distributions of uncertain parameters
is sparse and often not available. Since \gls{so} requires full probability distributions
of parameters, it is more sensitive to the assumptions made about 

\subsubsection{Conservativeness}

Although \gls{ro} problems solve problems with uncertainty,
\gls{ro} formulations result in solutions that are \emph{deterministically immune}
to all possible realizations of parameters in an uncertainty set~\cite{Bertsimas2011},
which is defined as conservativeness. There is extensive literature on \gls{ro} methods
that offer differing levels of conservativeness~\cite{Bertsimas2004}, especially
depending on the kind of uncertainty set considered.
\gls{so} formulations provide no guarantees of conservativity,
since the solution methods rely on randomized algorithms~\cite{Shmoys2004}.

\subsubsection{Stochasticity}

The solution of \gls{ro} problems is deterministic,
meaning that different instances of a design problem with
the same parameters will result in the same solution. This is not the case with \gls{so},
since the optimum depends on realizations of random variables.

It is important to highlight that,
although both \gls{ro} and \gls{so} seek to address the problem
of optimization under uncertainty, they solve fundamentally different problems. In an ideal world where
we have a problem that is tractable and globally optimal for both methods, the two different
approaches would result in different solutions.

\subsection{Geometric and signomial programming for engineering design}

Geometric programming\footnote{Programming refers to the mathematical formulation of an optimization problem.}
is a method of log-convex optimization that has been developed
to solve problems in engineering design~\cite{Duffin1967}. Although theory of the \gls{gp} has existed since
the 1960's, \gls{gp}s have recently experienced a resurgence due to the advent of polynomial-time
interior point methods~\cite{Nesterov1994} and improvements in computing. They have been
applied to a range of engineering design problems with success. For a non-exhaustive list of examples,
please refer to~\cite{Boyd2007}.

\gls{gp}s have been effective in aircraft conceptual design
(\cite{Hoburg2013},~\cite{Burton2017}).
However, the stringent mathematical requirements of a \gls{gp} limits its application to non-log-convex problems.
The \gls{sp} is the difference-of-log-convex extension of the \gls{gp} which can be applied to
solve this larger set of problems, albeit with the loss of some mathematical guarantees compared to the \gls{gp}~\cite{Kirschen2018}.
Aircraft pose some of the most challenging design problems~\cite{York2018}, and signomial programming
has been used to great effect in modeling and designing complex aircraft at a conceptual level quickly
and reliably as in \cite{York2018}, \cite{Kirschen2016} and \cite{Kirschen2018}.
Other interesting applications for SPs such as in network flow problems are being investigated.

Robust formulations exist for solving geometric programs with parametric uncertainty~\cite{Saab2018}.
The creation of a robust signomial programming framework to capture uncertainty in engineering
design, and specifically aircraft design, will allow us to have more confidence in the results
of the conceptual design phase, reduce program risk, and increase overall system performance.

\subsection{Contributions}

This paper proposes a tractable \gls{rsp} which we solve as a sequential \gls{rgp},
allowing us to implement robustness in non-log-convex problems such as aircraft design.
We extend the \gls{rgp} framework developed by Saab~\cite{Saab2018} to \gls{sp}s.
We implement the \gls{rsp} formulation on a simple aircraft design problem with several hundred
variables as defined in~\cite{Ozturk2018}.
The benefits of robust optimization are demonstrated both in ensuring design feasibility and performance
using \gls{mc} simulations of the uncertain parameters.
We further explore the benefits of \gls{ro} in multiobjective optimization, and propose
a goal programming \gls{rsp} formulation for risk minimization problems.



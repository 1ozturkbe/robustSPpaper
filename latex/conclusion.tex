\section{Conclusion}

In this paper we have motivated the use of robust optimization in conceptual engineering
design, in lieu of the mathematically non-rigorous methods of optimization under uncertainty
widely used in the aerospace industry today. We have developed a tractable \gls{rsp} formulation
in response to a need to optimize over uncertain parameters, extending
the tractable approximate RGP framework developed by Saab~\cite{Saab2018} to non-log-convex problems.
This \gls{rsp} formulation is a valuable contribution to the fields of robust
optimization and difference-of-convex programming.

\gls{rsp}s have a wide variety of potential applications in engineering design.
We demonstrated using a simple aircraft design problem
that using \gls{ro}, and specifically \gls{rsp}s in conceptual aircraft design will result in systems
that are more robust with respect to uncertainties in operational parameters,
such as payload mass and range, as well as uncertain environmental and manufacturing parameters.
By making designs immune to all realizations of uncertainty in a set, engineers can use
robust signomial programming to trade off
robustness and optimality within engineered systems in a tractable manner.

\gls{ro} has
the potential to change current aerospace design paradigms by introducing
mathematical rigor to design under uncertainty. Current aerospace
conceptual design practices still rely heavily on the expertise of established
engineers even in absence of prior experience exploring the design space.
\gls{rsp}s are compatible for use alongside physics based models
that are deprived of or lacking in data, and so can bring quantitative
measures of design reliability to the table and
steer the field of aerospace design towards physics-based tools and methods.

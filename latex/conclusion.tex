\section{Conclusion}

We have developed and applied a tractable \gls{rsp} formulation to a simple aircraft model,
and then discussed the benefits of having robust solutions. Our \gls{rsp} formulations extend
the tractable approximate RGP framework developed by Saab~\cite{Saab2018} to non-log-convex problems,
and are valuable contributions to the fields of robust optimization and difference-of-convex programming.

\gls{rsp}s have a wide variety of potential applications in engineering design.
We expect that using \gls{ro} in conceptual aircraft design will result in systems
that are more robust with respect to uncertainties in operational parameters,
such as payload mass and range, as well as uncertain environmental and manufacturing parameters.
By making designs immune to all realizations of uncertainty in a set, engineers can trade off
robustness and optimality within the context of an optimization framework in a tractable manner.

\gls{ro} has
the potential to change current aerospace design paradigms by introducing
mathematical rigor to design under uncertainty. Current aerospace
conceptual design practices still rely heavily on the expertise of established
engineers even in absence of prior experience exploring the design trade space.
\gls{ro} is compatible for use alongside physics based models
that are deprived of or lacking in data, and so can bring quantitative
measures of design reliability to the table and
steer the field of aerospace design towards physics-based tools and methods.



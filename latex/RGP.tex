\section{Robust Geometric Programming}
TODO by Ali: A Brief review of RGP
\begin{comment}
\subsection{Decoupled Form}
WOLG the objective function could be assumed linear (one dimensional variable) and
deprived of any data (epigraph formulation), therefore the cost will be ignored throughout this report.
Moreover, the GP will be represented as follows

\begin{equation}
\begin{aligned}
\textstyle{\sum}_{k=1}^{K_i}e^{\vec{a}_{ik}\vec{x} + b_{ik}} &\leq 1 &&\forall i \in \mathbf{P}\\
e^{\vec{a}_{i1}\vec{x} + b_{i1}} + e^{\vec{a}_{i2}\vec{x} + b_{i2}} &\leq 1 &&\forall i \in \mathbf{M}\\
e^{\vec{a}_{i1}\vec{x} + b_{i1}} &\leq 1 &&\forall i \in \mathbf{N}
\end{aligned}
\label{GP_decoupled}
\end{equation}

where
\begin{itemize}
\item $\mathbf{P}$ = $\left\{i : \vec{K}_i > 2\right\}$
\item $\mathbf{N}$ = $\left\{i : \vec{K}_i = 2\right\}$
\item $\mathbf{M}$ = $\left\{i : \vec{K}_i = 1\right\}$
\end{itemize}

%In words, the constraints are classified into three groups
%\begin{itemize}
% \item Three or more term posynomials which are referred to by the set $\mathbf{P}$
% \item Two term posynomials which are referred to by the set $\mathbf{N}$
% \item Monomials (one term posynomials) which are referred to by the set $\mathbf{M}$
%\end{itemize}
%\nomenclature{$\vec{t}$}{vector of artificial variables ($\textstyle{\sum_{i \in N} \vec{K}_i} \times 1$)}
%\nomenclature{$t_{ik}$}{the variable replacing the $k^{th}$ monomial in the $i^{th}$ posynomail}
The GP in equation \eqref{GP_decoupled} in convex form is represented as follows

\begin{equation}
\begin{aligned}
\log(\textstyle{\sum}_{k=1}^{K_i}e^{\vec{a}_{ik}\vec{x} + b_{ik}}) &\leq 0 &&\forall i \in \mathbf{P} \\
\log(e^{\vec{a}_{i1}\vec{x} + b_{i1}} + e^{\vec{a}_{i2}\vec{x} + b_{i2}}) &\leq 0 &&\forall i \in \mathbf{N} \\
\vec{a}_{i1}\vec{x} + b_{i1} &\leq 0 &&\forall i \in \mathbf{M}
\end{aligned}
\label{GP_convex}
\end{equation}
\subsection{Robust Counterpart}
In robust optimization, a formulation that is immune to the uncertainty in the system's
data should be derived. The data will be assumed living in an uncertainty set $\mathcal{U}$, where $\mathcal{U}$
is parameterized affinly by a perturbation vector $\vec{\zeta}$ as follows
\begin{equation}
\mathcal{U} = \left\{[\mat{A};\vec{b}] = [\mat{A}^0;\vec{b}^0] + \textstyle{\sum_{l=1}^{L}\zeta_l[\mat{A}^l;\vec{b}^l]}\right\}
\label{Data}
\end{equation}

where $\vec{\zeta}$ belongs to some perturbation set $\mathcal{Z} \in \mathbb{R}^L$ such that
\begin{equation}
\mathcal{Z} = \left\{ \vec{\zeta} \in \mathbb{R}^L: \exists \vec{u} \in \mathbb{R}^k:\mat{P}\vec{\zeta} + \mat{Q}\vec{u} + \vec{p} \in \textbf{K} \right\}
\label{perturbation_set}
\end{equation}

The robust counterpart of the uncertain geometric program given by equation \eqref{GP_decoupled} is

\begin{multicols}{2}
\begin{equation}
\begin{aligned}
\textstyle{\sum}_{k=1}^{K_i}e^{\vec{a}_{ik}\vec{x} + b_{ik}} &\leq 1 &&\forall i \in \mathbf{P} && \forall \vec{\zeta} \in \mathcal{Z}\\
e^{\vec{a}_{i1}\vec{x} + b_{i1}} + e^{\vec{a}_{i2}\vec{x} + b_{i2}} &\leq 1 &&\forall i \in \mathbf{M} && \forall \vec{\zeta} \in \mathcal{Z}\\
e^{\vec{a}_{i1}\vec{x} + b_{i1}} &\leq 1 &&\forall i \in \mathbf{N} && \forall \vec{\zeta} \in \mathcal{Z}.
\end{aligned}
\label{GP_counterparts}
\end{equation}\break
%$\iff$
%\break
\begin{equation}
\begin{aligned}
&\max_{\vec{\zeta} \in \mathcal{Z}} \left\{\textstyle{\sum}_{k=1}^{K_i}e^{\vec{a}_{ik}\vec{x} + b_{ik}}\right\} &&\leq 1 &&\forall i \in \mathbf{P}\\
&\max_{\vec{\zeta} \in \mathcal{Z}} \left\{e^{\vec{a}_{i1}\vec{x} + b_{i1}} + e^{\vec{a}_{i2}\vec{x} + b_{i2}}\right\} &&\leq 1 &&\forall i \in \mathbf{M}\\
&\max_{\vec{\zeta} \in \mathcal{Z}} \left\{e^{\vec{a}_{i1}\vec{x} + b_{i1}}\right\} &&\leq 1 &&\forall i \in \mathbf{N}
\end{aligned}
\label{GP_counterparts_finite}
\end{equation}
\end{multicols}

The above two sets of constraints state that the robust optimal solution should be
feasible for all possible realizations of the perturbation vector $\vec{\zeta}$.
Unfortunately, the robust counterpart of a geometric program is intractable using current solvers.
Throughout this report, an approximate formulation will be derived using robust linear programming.\\ [12pt]

\subsection{Conservative Tractable Formulation} \label{Conservative}
Since it is intractable to solve a robust geometric program,
this section will discuss a simple approximate formulation of the robust counterparts given by \eqref{GP_counterparts_finite}.\\[12pt]
The constraints corresponding to the set $\mathbf{M}$ of constraints are linear and therefore tractable, as a result, we should deal with the sets $\mathbf{P}$ and $\mathbf{N}$ only.\\[12pt]
The fact that $\max_{\vec{\zeta} \in \mathcal{Z}} \left\{\textstyle{\sum}_{k=1}^{K_i}e^{\vec{a}_{ik}\vec{x} + b_{ik}}\right\} \leq \sum_{k=1}^{K_i}\max_{\vec{\zeta} \in \mathcal{Z}} \left\{e^{\vec{a}_{ik}\vec{x} + b_{ik}}\right\}$ suggests the following safe constraints for the constraints corresponding to elements in $\mathbf{P}$ and $\mathbf{N}$
\begin{equation}
\sum_{k=1}^{K_i}\max_{\vec{\zeta} \in \mathcal{Z}} \left\{e^{\vec{a}_{ik}\vec{x} + b_{ik}}\right\} \leq 1
\label{conservative_robust_constraint}
\end{equation}
Equation \eqref{conservative_robust_constraint} suggests the following "conservative" formulation of the robust counterpart for the uncertain geometric program
\begin{equation}
\begin{aligned}
&\textstyle{\sum}_{k=1}^{K_i}\max_{\vec{\zeta} \in \mathcal{Z}} \left\{e^{\vec{a}_{ik}\vec{x} + b_{ik}}\right\} &&\leq 1 &&\forall i \in \mathbf{P},\mathbf{M}\\
&\max_{\vec{\zeta} \in \mathcal{Z}} \left\{e^{\vec{a}_{i1}\vec{x} + b_{i1}}\right\} &&\leq 1 &&\forall i \in \mathbf{N}
\end{aligned}
\label{GP_safe_conservative}
\end{equation}

which is equivalent to

\begin{equation}
\begin{aligned}
&\textstyle{\sum}_{k=1}^{K_i}t_{ik} &&\leq 1 &&\forall i \in \mathbf{P},\mathbf{M} \\
&\max_{\vec{\zeta} \in \mathcal{Z}} \left\{e^{\vec{a}_{ik}\vec{x} + b_{ik}}\right\} &&\leq t_{ik} &&\forall i \in \mathbf{P},\mathbf{M} &&\forall k \in 1,...,K_i\\
&\max_{\vec{\zeta} \in \mathcal{Z}} \left\{e^{\vec{a}_{i1}\vec{x} + b_{i1}}\right\} &&\leq 1 &&\forall i \in \mathbf{N}
\end{aligned}
\label{GP_safe_decoupled}
\end{equation}

In log-space, Equation \eqref{GP_safe_decoupled} is

\begin{equation}
\begin{aligned}
&\log(\textstyle{\sum}_{k=1}^{K_i}e^{s_{ik}}) &&\leq 0 &&\forall i \in \mathbf{P},\mathbf{M} \\
&\max_{\vec{\zeta} \in \mathcal{Z}} \left\{\vec{a}_{ik}\vec{x} + b_{ik}\right\} &&\leq s_{ik} &&\forall i \in \mathbf{P},\mathbf{M} &&\forall k \in 1,...,\vec{K}_i\\
&\max_{\vec{\zeta} \in \mathcal{Z}} \left\{\vec{a}_{i1}\vec{x} + b_{i1}\right\} &&\leq 0 &&\forall i \in \mathbf{N}
\end{aligned}
\label{GP_safe_convex}
\end{equation}

It can be observed that the convex constraints in equation \eqref{GP_safe_convex}
are deprived of data and uncertainty is only present in the linear constraints.
As a result, this problem is tractable using robust linear programming.\\[12pt]
Although this formulation might seem too conservative for some problems due to the fact that monomials are being decoupled,
however, it is exact for a wide range of problems that satisfy the following criteria
\begin{itemize}
\item $C_1$: The perturbation set is independent, e.g. $\mathcal{Z} = \left\{ \vec{\zeta} \in \mathbb{R}^L: \|\vec{\zeta}\|_{\infty} \leq \Gamma\right\}$
\item $C_2$: The monomials in each posynomial are independent, and if dependence only exists between the $'b'$s, then it should be "good" dependence, e.g. if $b_{11} = b_{11}^0 + \zeta_1$, $b_{12} = b_{12}^0 + \zeta_1 - \zeta_2$, and $b_{13} = b_{13}^0 + \zeta_2$, then $b_{11}$ and $b_{12}$ are dependent, but the dependence is good since the sign of the coefficients multiplied by the perturbation $\zeta_1$ are the same. However, $b_{12}$ and $b_{13}$ are dependent, but in a bad way due to the fact that the signs of the coefficients multiplied by the perturbation $\zeta_2$ are different
\end{itemize}
When $C_1$ and $C_2$ are satisfied for the $i^{th}$ posynomial, then
$$
\max_{\vec{\zeta} \in \mathcal{Z}} \left\{\textstyle{\sum}_{k=1}^{K_i}e^{\vec{a}_{ik}\vec{x} + b_{ik}}\right\} = \sum_{k=1}^{K_i}\max_{\vec{\zeta} \in \mathcal{Z}} \left\{e^{\vec{a}_{ik}\vec{x} + b_{ik}}\right\}
$$
and the "conservative" formulation is no longer conservative, but exact.\\

\subsection{Robust Two Term Posynomials} \label{twoTerm}
The focus now will be on modifying the methodology provided in section \ref{Conservative}
so that the solution is less conservative. This section will review Boyd's work on approximating two term
posynomials in log-space using piece-wise linear functions \footnote{ Kan-Lin Hsiung, Seung-Jean Kim, and Stephen Boyd.
Tractable approximate robust geometric programming. \textit{Optimization and Engineering}, 9(2):95118, Apr 2007}.\\[12pt]
Consider the convex function $\phi(x) = \log(1 + e^x)$, then the unique best r-term piece-wise linear convex lower approximation of $\phi$ is
\begin{equation}
\underline{\phi}_r =
\begin{cases}
0 \qquad &\text{if} \qquad x \in (- \infty, x_1]\\
\underline{a}_ix + \underline{b}_i &\text{if} \qquad x \in [x_i, x_{i+1}], i=1,2,..,r-2\\
x &\text{if} \qquad x \in [x_{r-1}, \infty)
\end{cases}
\label{lower_phi}
\end{equation}
such that
\begin{itemize}
\item $x_1 < x_2 < ... < x_{r-1}$
\item $\underline{a}_0 = 0 < \underline{a}_1 < \underline{a}_2 < ... < \underline{a}_{r-2} < \underline{a}_{r-1} = 1$
\item $\underline{a}_i + \underline{a}_{r-i-1} = 1\quad \forall i \in \left\{0,1, ..., r-1\right\}$
\item $\underline{b}_i = \underline{b}_{r-i-1} \quad \forall i \in \left\{1, ..., r-2\right\}$
\item $\underline{b}_0 = \underline{b}_{r-1} = 0$
\end{itemize}
Moreover, $\exists$ $\tilde{x}_1, \tilde{x}_2, ..., \tilde{x}_{r-2} \in \mathbf{R}$ satisfying
$$
x_1 < \tilde{x}_1 < x_2 < \tilde{x}_2 < ... < x_{r-2} < \tilde{x}_{r-2} < x_{r-1}
$$
such that $\underline{a}_ix + \underline{b}_i$ is tangent to $\phi$ at $\tilde{x}_i$.\\
Finally, the maximum approximation error $\epsilon_r$ of this piece-wise linearization occurs at the break points $x_1, ..., x_{r-1}$. The piece-wise linearization above will be used in approximating two term posynomials using piece-wise linear functions.\\[12pt]
Let $h= \log(e^{y_1} + e^{y_2})$ be a two term posynomial in log-space, where $y_1 = \vec{a}_1\vec{x} + b_1$ and $y_2 = \vec{a}_2\vec{x} + b_2$, then the unique best r-term piece-wise linear lower approximation is

\begin{equation}
\begin{aligned}
\underline{h_r} = \max \{&\underline{a}_{r-1}y_1 + \underline{a}_0y_2 + \underline{b}_0, \underline{a}_{r-2}y_1 + \underline{a}_1y_2 + \underline{b}_1, \underline{a}_{r-3}y_1 + \underline{a}_2y_2 + \underline{b}_2,\ ...,\\
& \underline{a}_{1}y_1 + \underline{a}_{r-2}y_2 + \underline{b}_{r-2}, \underline{a}_0y_1 + \underline{a}_{r-1}y_2 + \underline{b}_{r-1}\}
\end{aligned}
\end{equation}
while its unique best r-term piece-wise linear upper approximation is
\begin{equation}
\overline{h_r} = \underline{h_r} + \epsilon_r
\end{equation}
where $\underline{a}_{0}, \underline{a}_{1}, \underline{a}_{2}, ..., \underline{a}_{r-2}$ and $\underline{b}_{1}, \underline{b}_{2}, ..., \underline{b}_{r-2}, \underline{a}_{r-1}$ are as given in equation \eqref{lower_phi}, and $\epsilon_r$ is the maximum error between $\phi$ and $\underline{\phi}_r$.\\[12pt]
As a result, we know that $\overline{h}_r \geq h$, and therefore each posynomial in the set $\mathbf{N}$ is replaced by its best r-term piece-wise linear upper approximation (a safe approximation).\\[12pt]
This methodology needs to throw a sufficient number of constraints if a less conservative solution (better than the previous formulation) is to be achieved, however, it accounts for any dependency between the two monomials in a two term posynomial, and the problem will become tractable since a piece-wise linear constraint could be represented as a set of linear constraints.\\[12pt]

\subsection{Robust Large Term Posynomials}\label{k_term}
After taking care of the monomials and two term posynomials in a GP, its time to deal with the posynomial constraints corresponding to the set $\mathbf{P}$.\\[12pt]
The first step is to divide the large posynomial into smaller posynomials if possible. To do so, consider the set $\mathbf{I}_i$ associated with a large posynomial $p_i$, $i \in \mathbf{P}$ where
\begin{equation}
\mathbf{I}_i = \left\{ 1,2,...,K_i\right\}
\label{monomials_set}
\end{equation}
Then define an equivalence relation $\mathcal{R}$ on the set $\mathbf{I}_i$ where
\begin{equation}
\mathcal{R} = \left\{k_1 \sim k_2 \iff e^{\vec{a}_{ik_1}\vec{x} + b_{ik_1}} \text{ and } e^{\vec{a}_{ik_2}\vec{x} + b_{ik_2}} \text{ are directly or indirectly dependent} \right\}
\label{equivalence_relation}
\end{equation}
It is clear that $\mathcal{R}$ is an equivalence relation, therefore, when applied on $\mathbf{I}_i$, $\mathbf{I}_i$ would split into equivalence classes $S_{i,1}, S_{i,2}, ..., S_{i,N_e^i}$, $N_e^i \leq K_i$, then
$$
\max_{\vec{\zeta} \in \mathcal{Z}} \left\{\textstyle{\sum}_{k=1}^{K_i}e^{\vec{a}_{ik}\vec{x} + b_{ik}}\right\} = \textstyle{\sum}_{j=1}^{N_e^i} \max_{\vec{\zeta} \in \mathcal{Z}} \left\{\textstyle{\sum}_{k \in S_{i,j}}e^{\vec{a}_{ik}\vec{x} + b_{ik}}\right\}
$$
and therefore, the constraints corresponding to $\mathbf{P}$ in equation \eqref{GP_counterparts_finite} will be replaced by the following equivalent set of constraints
\begin{equation}
\begin{aligned}
\textstyle{\sum}_{j=1}^{N_e^i} t_{ij} &\leq 1 \qquad &&\forall i \in \mathbf{P}\\
\max_{\vec{\zeta} \in \mathcal{Z}} \left\{\textstyle{\sum}_{k \in S_{i,j}} e^{\vec{a}_{ik}\vec{x} + b_{ik}} \right\} &\leq t_{ij} &&\forall i \in \mathbf{P} \qquad &&\forall j = 1, ..., N_e^i\\
\end{aligned}
\label{equivalent_class_setP}
\end{equation}

Let
\begin{itemize}
\item $\mathbf{P_i'} = \left\{j:|S_{i,j}| \geq 3\right\}$ $\rightarrow$ intractable
\item $\mathbf{M_i'} = \left\{j:|S_{i,j}| = 2\right\}$ $\rightarrow$ tractable
\item $\mathbf{N_i'} = \left\{j:|S_{i,j}| = 1\right\}$ $\rightarrow$ tractable
\end{itemize}
Let $S_{i,j}^k$ be the $k^{th}$ element of $S_{i,j}$, and let $\mathcal{L}_{i,j}^k$ be the monomial $\vec{a}_{iS_{i,j}^k}\vec{x} + b_{iS_{i,j}^k}$, then \eqref{equivalent_class_setP} is equivalent to
\begin{equation}
\begin{aligned}
&\textstyle{\sum}_{j=1}^{N_e^i} t_{ij} &&\leq 1 \qquad &&\forall i \in \mathbf{P}\\
&\max_{\vec{\zeta} \in \mathcal{Z}} \left\{\textstyle{\sum}_{k=1}^{|S_{i,j}|} e^{\mathcal{L}_{i,j}^k} \right\} &&\leq t_{ij} &&\forall i \in \mathbf{P} \qquad &&\forall j \in \mathbf{P_i'}\\
&\max_{\vec{\zeta} \in \mathcal{Z}} \left\{e^{\mathcal{L}_{i,j}^1} + e^{\mathcal{L}_{i,j}^2} \right\} &&\leq t_{ij} &&\forall i \in \mathbf{P} \qquad &&\forall j \in \mathbf{M_i'}\\
&\max_{\vec{\zeta} \in \mathcal{Z}} \left\{e^{\mathcal{L}_{i,j}^1} \right\} &&\leq t_{ij} &&\forall i \in \mathbf{P} \qquad &&\forall j \in \mathbf{N_i'}
\end{aligned}
\label{equivalent_class_setP_separated}
\end{equation}

When most of the monomials are certain, or when the the perturbation set is independent, the large posynomials would certainly be reduced into several smaller posynomials, and in some cases into monomials.\\[12pt]
The discussion regarding large posynomial approximation will be divided into two parts
\begin{enumerate}
\item Robust large posynomials with uncertain coefficients $\vec{b}$ and certain exponents $\mat{A}$
\item Robust large posynomials with uncertain coefficients and exponents.
\end{enumerate}

\subsection{Uncertain Coefficients Only}
The exponents in many applications regarding geometric programming are certain, therefore it is interesting to look at the problem where only the $'b'$s are uncertain and are given by
$$
\vec{b} = \vec{b}^0 + \textstyle{\sum}_{l=1}^{L}\vec{b}^l\zeta_l
$$
where $\vec{\zeta} \in \mathcal{Z}$ as given by equation \eqref{perturbation_set}.\\
Consider the second set of constraints of equation \eqref{equivalent_class_setP_separated}, and note that
$$
\begin{aligned}
&\max_{\vec{\zeta} \in \mathcal{Z}} \left\{\textstyle{\sum}_{k=1}^{|S_{i,j}|} e^{\vec{a}_{iS_{i,j}^k}\vec{x} + b_{iS_{i,j}^k}} \right\} &&\leq t_{ij} &&\forall i \in \mathbf{P} \qquad &&\forall j \in \mathbf{P_i'}\\
\Leftrightarrow &\max_{\vec{\zeta} \in \mathcal{Z}} \left\{\textstyle{\sum}_{k=1}^{|S_{i,j}|} e^{\vec{a}_{iS_{i,j}^k}\vec{x} + b^0_{iS_{i,j}^k}}e^{\textstyle{\sum}_{l=1}^{L}b^l_{iS_{i,j}^k}\zeta_l} \right\} &&\leq t_{ij} &&\forall i \in \mathbf{P} \qquad &&\forall j \in \mathbf{P_i'}\\
\Leftrightarrow &\max_{\vec{\zeta} \in \mathcal{Z}} \left\{\textstyle{\sum}_{k=1}^{|S_{i,j}|}\textstyle{\prod}_{l=1}^{L}e^{b^l_{iS_{i,j}^k}\zeta_l} e^{\vec{a}_{iS_j^i(k)}\vec{x} + b^0_{iS_{i,j}^k}} \right\} &&\leq t_{ij} &&\forall i \in \mathbf{P} \qquad &&\forall j \in \mathbf{P_i'}\\
\end{aligned}
$$
The above constraint could be thought of as an uncertain linear constraint in terms of the variables $v_{i,j}^k = e^{\vec{a}_{iS_{i,j}^k}\vec{x} + b^0_{iS_{i,j}^k}}$ for $k = 1,...,|S_{i,j}|$ as follows
\begin{equation}
\max_{\vec{\zeta} \in \mathcal{Z}} \left\{\textstyle{\sum}_{k=1}^{|S_{i,j}|}\left(\textstyle{\prod}_{l=1}^{L}e^{b^l_{iS_{i,j}^k}\zeta_l}\right) v_{i,j}^k \right\} \leq t_{ij} \qquad \forall i \in \mathbf{P} \qquad \forall j \in \mathbf{P_i'}
\label{linearCon_expPerts}
\end{equation}
Although the constraints are linear in terms of $\vec{v_j^i}$, the perturbations are not affine but exponential. And since it is hard to deal with exponential perturbations, then it might be more convenient to linearize the perturbations.

\subsubsection{Linearizing perturbations}
Although the perturbations are not affine, however, they have some nice property which is convexity. This implies that there exists some half-space (affine function) $[\vec{f}_{i,j}^k]^T\vec{\zeta} + g_{i,j}^k \geq \textstyle{\prod}_{l=1}^{L}e^{b^l_{iS_{i,j}^k}\zeta_l}$. Therefore, a safe approximation of the constraints in equation \eqref{linearCon_expPerts} is
\begin{equation}
\max_{\vec{\zeta} \in \mathcal{Z}} \left\{\textstyle{\sum}_{k=1}^{|S_{i,j}|}\left([\vec{f}_{i,j}^k]^T\vec{\zeta}\right)v_{i,j}^k \right\} + \textstyle{\sum}_{k=1}^{|S_{i,j}|}g_{i,j}^k v_{i,j}^k \leq t_{ij} \qquad \forall i \in \mathbf{P} \qquad \forall j \in \mathbf{P_i'}
\label{linearCon_linPerts}
\end{equation}
To construct the half-space $[\vec{f}_{i,j}^k]^T\vec{\zeta} + g_{i,j}^k$, and taking into consideration the fact that $-1 \leq \zeta_l \leq 1$ for $l = 1,...,L$, it is suggested to follow the steps listed below
\begin{enumerate}
\item Find the list of vertices $\mathcal{V}$ of the unit box in $\mathbf{R}^L$
\item Find the list of values $\mathcal{O}$ of $\textstyle{\prod}_{l=1}^{L}e^{b^l_{iS_{i,j}^k}\zeta_l}$ at the vertices $\mathcal{V}$, note that the $i^{th}$ vertex corresponds to the $i^{th}$ value.
\item Find the maximum $M_k$ and minimum $m_k$ of $\mathcal{O}$ and their corresponding vertices $\vec{\zeta}_M$ and $\vec{\zeta}_m$.
\item Solve the least-square problem $min \sqrt{\textstyle{\sum}_{\alpha=1}^{|\mathcal{O}|}([\vec{f}_{i,j}^k]^T\mathcal{V}_{\alpha} + g_{i,j}^k - \mathcal{O}_{\alpha})^2}$ such that $[\vec{f}_{i,j}^k]^T\vec{\zeta}_M + g_{i,j}^k = M_k$ and $[\vec{f}_{i,j}^k]^T\vec{\zeta}_m + g_{i,j}^k = m_k$
\end{enumerate}
The half-space constructed above is a safe approximation of the intractable large posynomial.

\subsubsection{SP compatible constraint}
By looking again at equation \eqref{linearCon_linPerts}, it can be seen that it is now tractable using robust linear programming. Unfortunately, The resulting set of robust constraints is not always GP compatible due to the fact that some components of $\vec{f}_{i,j}^k$ might not be positive, as a result, an SP need to be solved.\\
The solution of the SP depends on some initial guess, and a global optimum is not always guaranteed, therefore, it is important to choose a good initial guess.\\

\subsection{Uncertain Coefficients and Exponents}
Things would become harder when the exponents are also uncertain. Consider again the set of constraints

\begin{equation}
\max_{\vec{\zeta} \in \mathcal{Z}} \left\{\textstyle{\sum}_{k=1}^{|S_j^i|} e^{\vec{a}_{iS_{i,j}^k}\vec{x} + b_{iS_{i,j}^k}} \right\} \leq t_{ij} \qquad \forall i \in \mathbf{P} \qquad \forall j \in \mathbf{P_i'}
\label{second_set}
\end{equation}

Let $\phi_{i,j}(s) : S_{i,j} \rightarrow S_{i,j}$ be a bijection from $S_{i,j}$ onto $S_{i,j}$ known as a permutation on $S_{i,j}$. Note that \eqref{second_set} is equivalent to
\begin{equation}
\max_{\vec{\zeta} \in \mathcal{Z}} \left\{\textstyle{\sum}_{k=1}^{|S_{i,j}|} e^{\vec{a}_{iS_{i,j}^{\phi_{i,j}(k)}}\vec{x} + b_{iS_{i,j}^{\phi_{i,j}(k)}}} \right\} \leq t_{ij} \qquad \forall i \in \mathbf{P} \qquad \forall j \in \mathbf{P_i'}
\label{permutation_k_term}
\end{equation}
The suggested methodology is to maximize each two monomials alone starting from the fact that
$$
\max \left\{ a + b + c + d\right\} \leq \max \left\{a + b\right\} + \max \left\{c + d\right\}
$$
Let $\vec{a}_{iS_{i,j}^{\phi_{i,j}(k)}}\vec{x} + b_{iS_{i,j}^{\phi_{i,j}(k)}} = \mathcal{L}^{\phi(k)}_{i,j}$, and Assume $|S_{i,j}|$ is even, then
$$
\max_{\vec{\zeta} \in \mathcal{Z}} \left\{\textstyle{\sum}_{k=1}^{|S_{i,j}|} e^{\mathcal{L}^{\phi(k)}_{i,j}}\right\} \leq \textstyle{\sum}_{k=1}^{|S_{i,j}|/2} \max_{\vec{\zeta} \in \mathcal{Z}} \left\{e^{\mathcal{L}^{\phi(2(k-1)+1)}_{i,j}} + e^{\mathcal{L}^{\phi(2k)}_{i,j}}\right\}
$$
for any permutation $\phi_{i,j}$ in the set of permutation functions $\mathcal{P}_{i,j}$, Therefore
$$
\max_{\vec{\zeta} \in \mathcal{Z}} \left\{\textstyle{\sum}_{k=1}^{|S_{i,j}|} e^{\mathcal{L}^{\phi(k)}_{i,j}}\right\} \leq \min_{\phi \in \mathcal{P}_{i,j}}\left\{\textstyle{\sum}_{k=1}^{|S_{i,j}|/2} \max_{\vec{\zeta} \in \mathcal{Z}} \left\{e^{\mathcal{L}^{\phi(2(k-1)+1)}_{i,j}} + e^{\mathcal{L}^{\phi(2k)}_{i,j}}\right\}\right\}
$$
and

\begin{equation}
\min_{\phi \in \mathcal{P}_{i,j}}\left\{\textstyle{\sum}_{k=1}^{|S_{i,j}|/2} \max_{\vec{\zeta} \in \mathcal{Z}} \left\{e^{\mathcal{L}^{\phi(2(k-1)+1)}_{i,j}} + e^{\mathcal{L}^{\phi(2k)}_{i,j}}\right\}\right\} \leq t_{ij} \qquad \forall i \in \mathbf{P} \qquad \forall j \in \mathbf{P_i'}
\label{min_safe_app}
\end{equation}

is a safe approximation of \eqref{permutation_k_term}.\\[12pt]
The size of the permutations set $\mathcal{P}_{i,j}$ is quit large for large posynomials, however, due to the fact that
$$
max \left\{a + b\right\} + max \left\{c + d\right\} = max \left\{c + d\right\} + max \left\{a + b\right\} = max \left\{b + a\right\} + max \left\{d + c\right\}
$$
Then some perturbations are similar in terms of the resulting safe approximation, and the permutation set could be modified to contain the ``different" permutations only. Indeed, from now on, $\mathcal{P}_{i,j}$ will represent the set of ``different'' permutations, and not all permutations.\\[12pt]
It is hard to find the permutation that minimizes the above expression due to the large number of possible combinations of permutations, therefore, let $\hat{\mathcal{P}}_{i,j}$ be a subset of $\mathcal{P}_{i,j}$, where the permutations are either chosen depending on the structure of the posynomial, or randomly selected. Also, note that the cardinality of $\hat{\mathcal{P}}_{i,j}$ depends on the size of the problem, and should not increase as fast as $\mathcal{P}_{i,j}$.\\
Using the above sets, the following lemma will be utilized to find a relatively good permutation.

\begin{lemma}
Consider the two optimization problems
$$
\begin{aligned}
&\min f(\vec{x})\\
&\text{s.t. } \mathcal{S}_i(\vec{x}) \leq 0 \qquad i = 1,2,...,n
\end{aligned}
$$
and
$$
\begin{aligned}
&\min f(\vec{x})\\
&\text{s.t. } \mathcal{T}_i(\vec{x}) \leq 0 \qquad i = 1,2,...,n
\end{aligned}
$$
and let $\vec{x}_1$ and $\vec{x}_2$ be the optimal solutions of the first and second optimization problems respectively.\\
If $\mathcal{T}_i(\vec{x}_1) \leq \mathcal{S}_i(\vec{x}_1) \quad \forall i \in \left\{1,2,...,n\right\}$, then $f(\vec{x}_2) \leq f(\vec{x}_1)$
\end{lemma}
\begin{proof}
$$\mathcal{T}_i(\vec{x}_1) \leq \mathcal{S}_i(\vec{x}_1) \leq 0 \quad \forall i \in \left\{1,2,...,n\right\} $$
Then $\vec{x}_1$ is a feasible solution for the second optimization problem.\\
$\vec{x}_2$ is the optimal solution of the second problem, then $$f(\vec{x}_2) \leq f(\vec{x}_{feasible}) \implies f(\vec{x}_2) \leq f(\vec{x}_1)$$
\end{proof}

Finding the least conservative solution can be done by solving a sequence of geometric programs.\\
Start by replacing the set of large posynomial constraints by
\begin{equation}
\textstyle{\sum}_{k=1}^{|S_{i,j}|/2} \max_{\vec{\zeta} \in \mathcal{Z}} \left\{e^{\mathcal{L}^{\phi(2(k-1)+1)}_{i,j}} + e^{\mathcal{L}^{\phi(2k)}_{i,j}}\right\}\leq t_{ij} \qquad \forall i \in \mathbf{P} \qquad \forall j \in \mathbf{P_i'}
\label{two_term_max_even}
\end{equation}
where $\phi_{i,j}$ is randomly selected from $\hat{\mathcal{P}}_{i,j}$.\\
\eqref{two_term_max_even} is a safe approximation for \eqref{permutation_k_term}, and is equivalent to
\begin{equation}
\begin{aligned}
&\textstyle{\sum}_{k=1}^{|S_{i,j}|/2} z_{ij}^k &&\leq t_{ij} \qquad &&\forall i \in \mathbf{P} \qquad &&\forall j \in \mathbf{P_i'}\\
&\max_{\vec{\zeta} \in \mathcal{Z}} \left\{e^{\mathcal{L}^{\phi(2(k-1)+1)}_{i,j}} + e^{\mathcal{L}^{\phi(2k)}_{i,j}}\right\} &&\leq z_{ij}^k &&\forall i \in \mathbf{P} \qquad &&\forall j \in \mathbf{P_i'} &&\forall k \in \left\{1,2..,|S_{i,j}|/2\right\}
\end{aligned}
\label{two_term_even}
\end{equation}

Summing up all the work from different sections, it can be seen that all the constraints are now tractable using linear programming techniques.\\
The new formulation is either composed of monomials, two-term posynomials, or data-deprived large posynomials. As a result, this formulation is tractable using our knowledge from the previous sections.\\[12pt]
The following algorithm illustrates how the permutations are chosen:
\begin{enumerate}
\item randomly choose the permutations $\phi_{i,j}$ for the set $\hat{\mathcal{P}}$
\item solve the new formulated optimization problem and let $\vec{x}_1$ be the solution
\item repeat
\begin{enumerate}
\item select the new permutations $\phi_{i,j} \in \hat{\mathcal{P}}_{i,j}$ such that $\phi_{i,j}$ minimizes $\textstyle{\sum}_{k=1}^{|S_{i,j}|/2} \max_{\vec{\zeta} \in \mathcal{Z}} \left\{e^{\mathcal{L}^{\phi(2(k-1)+1)}_{i,j}} + e^{\mathcal{L}^{\phi(2k)}_{i,j}}\right\}\bigg\rvert_{\vec{x}_{i-1}}$
\item solve the optimization problem again with the new permutations and let $\vec{x}_i$ be the solution
\item if $\vec{x}_i = \vec{x}_{i-1}$ : break
\end{enumerate}
\end{enumerate}
Although the solution might not be the least conservative, however, the solution is relatively good.\\[12pt]
We are now ready to solve a robust geometric program approximately, the next section will discuss extending the RGP into RSP.
\end{comment}
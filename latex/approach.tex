\section{Approach to Solving Robust Signomial Programs}

This section presents a heuristic algorithm to safely solve a \gls{rsp}
based on our previous discussion on robust geometric programming.

\subsection{General \gls{rsp} Solver}
As mentioned before, a common heuristic algorithm to solve an \gls{sp} is
by sequentially solving local \gls{gp} approximations, but the solution is not guaranteed
to be globally optimal. Our approach to solve an \gls{rsp} is based on sequentially solving
local \gls{rgp} approximations. Below we provide a step-by-step algorithm to solve an \gls{rsp}:

\begin{enumerate}
    \item Choose an initial guess $x_0$
    \item Repeat
    \begin{enumerate}
        \item Find the local GP approximation of the \gls{sp} at $x_i$.
        \item Find the RGP formulation of the GP.
        \item Solve the RGP to obtain $x_{i+1}$.
        \item If $x_{i+1} \approx x_{i}$: break
    \end{enumerate}
\end{enumerate}

Similar to an \gls{sp}, a good initial guess would lead to faster convergence and possibly a better solution.
A quick candidate is the deterministic solution of the uncertain \gls{sp}, which will certainly lead to a faster convergence.

\begin{figure}
    \begin{center}
    \begin{tikzpicture}[auto, align=center, text width=2.75cm, scale = 0.9]
        \begin{scope}[node distance=2cm]
        \node[block, name=detSP] at (0,0) (detSP) {Solve deterministic SP};
        \node[block, name=localGP] at (4,0) (localGP) {Make local GP approximation};
        \node[block, name=localRGP] at (10,0) (localRGP) {Formulate local RGP};
        \node[block, name=solveRGP] at (14,0) (solveRGP) {Solve local RGP};
        \node[block, name=xi] at (14,-2) (xi) {$\Delta x \leq reltol$?};
        \node[block, name=solution] at (14,-4) (solution) {Solution $x_{n}$};

        \draw[->] (detSP) -- node[name=x0] {$x_0$} (localGP);
        \draw[->] (localGP) -- node[name=choosemethod] {Choose methodology} (localRGP);
        \draw[->] (localRGP) -- (solveRGP);
        \draw[->] (solveRGP) -- (xi);
        \draw[->] (xi) -| node[name=no] {No, $x_{i+1} = x_i.$} (localGP);
        \draw[->] (xi) -- node[name=yes] {Yes.} (solution);
        \end{scope}
    \end{tikzpicture}
    \caption{A block diagram showing the steps of solving an \gls{rsp}.}
        \label{fig:rspsolve}
\end{center}
\end{figure}

Any of the previously mentioned methodologies can be used to formulate the local RGP approximation. 
However, depending on the RGP formulation chosen to solve an \gls{rsp}, the last formulation and solution
blocks in Figure \ref{fig:rspsolve} are tweaked slightly for a faster convergence.
\ \\
\subsection{Best Pairs \gls{rsp} Solver}

If the Best Pairs methodology is exploited, then the above algorithm would change so that
each iteration would solve the local RGP approximation and choose the best permutation
for each large posynomial. The modified algorithm would become as follows:

\begin{enumerate}
    \item Choose an initial guess $x_0$
    \item Repeat
    \begin{enumerate}
        \item Find the local GP approximation of the SP at $x_i$.
        \item For each large posynomial constraint, select the new permutation $\phi$
                such that $\phi$ minimizes the robust large constraint evaluated at $x_i$.
        \item Solve the approximate tractable counterparts of the local \gls{gp} in
                \eqref{GP_counterparts_finite}, and let $\vec{x}_{i+1}$ be the solution.
        \item If $x_{i+1} \approx x_{i}$: break.
    \end{enumerate}
\end{enumerate}

\subsection{Linearized Perturbations \gls{rsp} Solver}

On the other hand, if the Linearized Perturbations formulation is to be used,
then we can avoid solving a \gls{sp} at each iteration by first
approximating the original \gls{sp} constraints locally, and in the same loop approximating
the robustified possibly signomial constraints locally, thus solving a
\gls{gp} at each iteration instead of an \gls{sp}. The algorithm would then become as follows:

\begin{enumerate}
    \item Choose an initial guess $x_0$.
    \item Repeat:
    \begin{enumerate}
        \item Find the local GP approximation of the SP at $x_i$.
        \item Robustify the constraints of the local GP approximation using the Linearized Perturbations methodology.
        \item Find the local GP approximation of the resulting local SP at $x_i$.
        \item Solve the local GP approximation in step c to obtain $x_{i+1}$
        \item If $x_{i+1} \approx x_{i}$: break.
    \end{enumerate}
\end{enumerate}

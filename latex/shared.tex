\title{Optimal Aircraft Design Decisions under Uncertainty via Robust Signomial Programming}

\author{Berk Ozturk \footnote{PhD Candidate, Department of Aeronautics and Astronautics, AIAA Member.}
and Ali Saab\footnote{S.M. 2018, Department of Aeronautics and Astronautics.}}
\affil{Massachusetts Institute of Technology, Cambridge, MA, 02139}

% Data used by 'handcarry' option if invoked
% \AIAApapernumber{2019}
% \AIAAconference{Conference Name, Date, and Location}
% \AIAAcopyright{\AIAAcopyrightD{2019}}

% Define commands to assure consistent treatment throughout document
\newcommand{\eqnref}[1]{(\ref{#1})}
\newcommand{\class}[1]{\texttt{#1}}
\newcommand{\package}[1]{\texttt{#1}}
\newcommand{\file}[1]{\texttt{#1}}
\newcommand{\BibTeX}{\textsc{Bib}\TeX}

\renewcommand{\vec}{\mathbf}
\newcommand{\mat}{\mathbf}

\usepackage[utf8]{inputenc}
\usepackage{bbm}
\usepackage{booktabs}
\usepackage{amsmath}
\usepackage{csvsimple}
\usepackage{filecontents}
\usepackage{float}
\usepackage[acronyms]{glossaries}
\usepackage[sort&compress, numbers]{natbib}
\usepackage{tabularx}
\usepackage[toc,page]{appendix}
\usepackage{tikz}
\usepackage{tkz-kiviat}
\usetikzlibrary{shapes, arrows, positioning, decorations.markings, fit}
\usepackage{pgfplotstable}
\usepackage{subcaption}
\usepackage{graphicx}
\makeglossaries

\tikzstyle{block} = [draw, rectangle,
minimum height=3em, minimum width=6em]
\tikzstyle{sum} = [draw, circle]
\tikzstyle{input} = [coordinate]
\tikzstyle{output} = [coordinate]
\tikzstyle{pinstyle} = [pin edge={to-,thin,black}]
\tikzstyle{vecArrow} = [thick, decoration={markings,mark=at position
1 with {\arrow[semithick]{open triangle 60}}},
double distance=1.4pt, shorten >= 5.5pt,
preaction = {decorate},
postaction = {draw,line width=1.4pt, white,shorten >= 4.5pt}]
\tikzstyle{innerWhite} = [semithick, white,line width=1.4pt, shorten >= 4.5pt]

\makeatletter
\csvset{
autotabularcenter/.style={
file=#1,
after head=\csv@pretable\begin{tabular}{|*{\csv@columncount}{c|}}
                            \csv@tablehead,
                            table head=\hline\csvlinetotablerow\\\hline,
                            late after line=\\,
                            table foot=\\\hline,
                            late after last line=\csv@tablefoot
\end{tabular}\csv@posttable,
command=\csvlinetotablerow},
autobooktabularcenter/.style={
file=#1,
after head=\csv@pretable\begin{tabular}{*{\csv@columncount}{c}}
                            \csv@tablehead,
                            table head=\toprule\csvlinetotablerow\\\midrule,
                            late after line=\\,
                            table foot=\\\bottomrule,
                            late after last line=\csv@tablefoot
\end{tabular}\csv@posttable,
command=\csvlinetotablerow},
}
\makeatother
\newcommand{\csvautotabularcenter}[2][]{\csvloop{autotabularcenter={#2},#1}}
\newcommand{\csvautobooktabularcenter}[2][]{\csvloop{autobooktabularcenter={#2},#1}}
\newcommand{\norm}[1]{\left\lVert#1\right\rVert}

\newacronym{bsfc}{BSFC}{brake specific fuel consumption}
\newacronym{ceg}{CEG}{Convex Engineering Group}
\newacronym{gp}{GP}{Geometric Program}
\newacronym{mc}{MC}{Monte Carlo}
\newacronym{sp}{SP}{Signomial Program}
\newacronym{rgp}{RGP}{Robust Geometric Program}
\newacronym{rsp}{RSP}{Robust Signomial Program}
\newacronym{ro}{RO}{Robust Optimization}
\newacronym{so}{SO}{Stochastic Optimization}
\newacronym{mdo}{MDO}{Multidisciplinary Design Optimization}
\newacronym{dc}{DC}{difference-of-convex}
\newacronym{nlp}{NLP}{Nonlinear Program}
\newacronym{rhs}{RHS}{right hand side}
\newacronym{lhs}{LHS}{left hand side}
\newacronym{cv}{CV}{coefficient of variation}

\newcommand{\AR}{A\!R}
\newcommand{\BSFC}{{\rm B\!S\!F\!C}}
\newcommand{\CDA}{C\!D\!A}
\newcommand{\RC}{{\rm R\!C}}

\documentclass[journal]{new-aiaa}

 \title{Optimal Design Decisions via Robust Signomial Programming}

 \author{
  A. Saab%
    \thanks{Job Title, Department, Address, and AIAA Member Grade.}
  \ and B. Ozturk\thanksibid{1}\\
  {\normalsize\itshape
   Business or Academic Affiliation, City, Province, Zipcode, Country}\\
  % \and
  % Third E. Burnell%
  %  \thanks{Job Title, Department, Address, and AIAA Member Grade.}\\
  % {\normalsize\itshape
  % Business or Academic Affiliation, City, Province, Zipcode, Country}
  % \and
  % Fourth W. Hoburg%
  % \thanks{Job Title, Department, Address, and AIAA Member Grade.}\\
  % {\normalsize\itshape
  % Business or Academic Affiliation, City, Province, Zipcode, Country}
 }

 % Data used by 'handcarry' option if invoked
 \AIAApapernumber{YEAR-NUMBER}
 \AIAAconference{Conference Name, Date, and Location}
 \AIAAcopyright{\AIAAcopyrightD{YEAR}}

 % Define commands to assure consistent treatment throughout document
 \newcommand{\eqnref}[1]{(\ref{#1})}
 \newcommand{\class}[1]{\texttt{#1}}
 \newcommand{\package}[1]{\texttt{#1}}
 \newcommand{\file}[1]{\texttt{#1}}
 \newcommand{\BibTeX}{\textsc{Bib}\TeX}

\renewcommand{\vec}{\mathbf}
\newcommand{\mat}{\mathbf}

\usepackage[utf8]{inputenc}
\usepackage{algorithm}
\usepackage{bbm}
\usepackage{amsmath}
\usepackage{amsthm}
\usepackage{amssymb}
\usepackage{multicol}
\usepackage{tabularx}
\usepackage[toc,page]{appendix}
\usepackage{verbatim}
\usepackage{tikz}
\usepackage{tkz-kiviat}
\usetikzlibrary{arrows}
 
 \newtheorem{theorem}{Theorem}[section]
 \newtheorem{corollary}{Corollary}[theorem]
 \newtheorem{lemma}[theorem]{Lemma}
 \newtheorem{proposition}[theorem]{Proposition}

\title{Implications of robustness for geometry}
\author{Berk {\"O}zt{\"u}rk\footnote{PhD Candidate, Department of Aeronautics and Astronautics.}}
\affil{Massachusetts Institute of Technology, Cambridge, MA, 02139}

\begin{document}

\maketitle

\section*{Nomenclature}

\begin{tabbing}
  XXXXX \= \kill% this line sets tab stop
  CEG \> Convex Engineering Group \\
  GP \> geometric program \\
  LHS \> left hand side \\
  MDO \> multidisciplinary design optimization \\
  NLP \> nonlinear program \\
  SP \> signomial program \\
  RGP \> robust geometric program \\
  RHS \> right hand side \\
  RO \> robust optimization \\
  RSP \> robust signomial program \\
  SO \> stochastic optimization \\
 \end{tabbing}


\printglossary

\section{Mathematical ideas behind robust optimization}

RO is a tractable method for optimization under uncertainty, and specifically under uncertain
parameters. It optimizes the worst-case objective outcome over uncertainty sets,
unlike general stochastic optimization methods which optimize statistics of the distribution
of the objective over probability distributions of uncertain parameters. As such, RO
sacrifices generality for tractability, probabilistic guarantees and engineering intuition. The following
descriptions have been paraphrased from~\cite{Ozturk2019}.

Given a general optimization problem under parametric uncertainty, we define the set of possible
realizations of uncertain vector of parameters $u$ in the uncertainty set $\mathcal{U}$. This
allows us to define the problem under uncertainty below.
\begin{align*}
    \text{min} &~~f_0(x) \\
    \text{s.t.}     &~~f_i(x,u) \leq 0,~\forall u \in \mathcal{U},~i = 1,\ldots,n
\end{align*}
This problem is infinite-dimensional, since it is possible to formulate an infinite number of constraints
with the countably infinite number of possible realizations of $u \in \mathcal{U}$. To circumvent this issue,
we can define the following robust formulation of the uncertain problem below.
\begin{align*}
    \text{min} &~~f_0(x) \\
    \text{s.t.}     &~~\underset{u \in \mathcal{U}}{\text{max}}~f_i(x,u) \leq 0,~i = 1,\ldots,n
\end{align*}
This formulation hedges against the worst-case realization of the uncertainty in the defined uncertainty
set. The set is often described by a norm, which contains possible uncertain outcomes from distributions with
bounded support
\begin{equation}
    \begin{split}
        \text{min} &~~f_0(x) \\
    \text{s.t.}     &~~\underset{u}{\text{max}}~f_i(x,u) \leq 0,~i = 1,\ldots,n \\
                    &~~\left\lVert u \right\rVert \leq \Gamma \\
        \end{split}
    \label{eq:normform}
\end{equation}
where $\Gamma$ is defined by the user as a global uncertainty bound. The larger the $\Gamma$,
the greater the size of the uncertainty set that is protected against.

\subsection{Clarification of nomenclature}

In the context of optimization, the inputs to a design problem are referred to as parameters,
and the outputs are the objective and design variables.
This can be confusing since ESP refers to its driving variables as parameters,
which are actually the design variables of the original problem.
Since optimization here is occurring prior to geometry generation,
I will refer to optimization inputs as parameters, optimization outputs
as design variables.

\section{Role of sensitivities}

As a result of the primal-dual interior point solution method,
GPkit returns the sensitivities of the objective function to each parameter.

\begin{equation}

\end{equation}

Note that this is the total derivative, which means that other variables are
also free to change when the parameter is perturbed.

The partial derivate implies tha


\section{Geometric representations of the Jacobian}

The total derivatives of design variables with respect to input parameters
gives us an idea about the sensitivity of the geometry to perturbations in the parameters.
Geometry tools (eg. ESP) already have methods to handle the sensitivity of the geometry

\section{Goals}

The goals of this framework will be to:
\begin{itemize}
    \item \textbf{}
\end{itemize}

\bibliographystyle{new-aiaa}
\bibliography{main}

\end{document}

% - Release $Name:  $ -

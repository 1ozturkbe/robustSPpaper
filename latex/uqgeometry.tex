\documentclass[journal]{new-aiaa}

 \title{Optimal Design Decisions via Robust Signomial Programming}

 \author{
  A. Saab%
    \thanks{Job Title, Department, Address, and AIAA Member Grade.}
  \ and B. Ozturk\thanksibid{1}\\
  {\normalsize\itshape
   Business or Academic Affiliation, City, Province, Zipcode, Country}\\
  % \and
  % Third E. Burnell%
  %  \thanks{Job Title, Department, Address, and AIAA Member Grade.}\\
  % {\normalsize\itshape
  % Business or Academic Affiliation, City, Province, Zipcode, Country}
  % \and
  % Fourth W. Hoburg%
  % \thanks{Job Title, Department, Address, and AIAA Member Grade.}\\
  % {\normalsize\itshape
  % Business or Academic Affiliation, City, Province, Zipcode, Country}
 }

 % Data used by 'handcarry' option if invoked
 \AIAApapernumber{YEAR-NUMBER}
 \AIAAconference{Conference Name, Date, and Location}
 \AIAAcopyright{\AIAAcopyrightD{YEAR}}

 % Define commands to assure consistent treatment throughout document
 \newcommand{\eqnref}[1]{(\ref{#1})}
 \newcommand{\class}[1]{\texttt{#1}}
 \newcommand{\package}[1]{\texttt{#1}}
 \newcommand{\file}[1]{\texttt{#1}}
 \newcommand{\BibTeX}{\textsc{Bib}\TeX}

\renewcommand{\vec}{\mathbf}
\newcommand{\mat}{\mathbf}

\usepackage[utf8]{inputenc}
\usepackage{algorithm}
\usepackage{bbm}
\usepackage{amsmath}
\usepackage{amsthm}
\usepackage{amssymb}
\usepackage{multicol}
\usepackage{tabularx}
\usepackage[toc,page]{appendix}
\usepackage{verbatim}
\usepackage{tikz}
\usepackage{tkz-kiviat}
\usetikzlibrary{arrows}
 
 \newtheorem{theorem}{Theorem}[section]
 \newtheorem{corollary}{Corollary}[theorem]
 \newtheorem{lemma}[theorem]{Lemma}
 \newtheorem{proposition}[theorem]{Proposition}

\title{Implications of robustness to uncertainty for design geometry}
\author{Berk {\"O}zt{\"u}rk\footnote{PhD Candidate, Department of Aeronautics and Astronautics.}}
\affil{Massachusetts Institute of Technology, Cambridge, MA, 02139}

\begin{document}

\maketitle

\section*{Nomenclature}

\begin{tabbing}
  XXXXX \= \kill% this line sets tab stop
  GP \> geometric program \\
  MDAO \> multidisciplinary analysis and optimization \\
  MDO \> multidisciplinary design optimization \\
  SP \> signomial program \\
  RGP \> robust geometric program \\
  RO \> robust optimization \\
  RSP \> robust signomial program \\
  SO \> stochastic optimization \\
  UQ \> uncertainty quantification \\ 
 \end{tabbing}
\printglossary

\section{Motivation}

Geometry is a key component of the design process. Most obviously, it translates values of design variables
through a configuration to allow engineers to visualize and compare designs. But it also
serves other critical purposes in design, most notably in analysis and manufacturing.
However, a lot of information is lost between different phases of
the design process as different tools 
require varying levels of detail or fidelity. 
There have been attempts to have a seamless relationship between geometry and design,
especially through our initiatives to
create a GPkit-ESP integration. We call this `continuum design', where a single representation
can serve as a common medium to explain design trade-offs, but can also be used to bring
high-fidelity analysis tools and manufacturing considerations into the mix.

In parallel to these efforts, we have been developing optimization tools
that provide designs that are robust to uncertainty in
problem parameters. Optimization under uncertainty has been underutilized for aerospace
system design for several reasons. Uncertainty propagation in general \gls{mdao}
methods, otherwise known as stochastic optimization, is computationally costly and
generally intractable. This is unfortunately the only form of uncertainty propagation
available for most conceptual aerospace design tools
since they have black-box analyses\cite{Yao2011}.
For design problems with explicit constraints,
we have proposed robust optimization in a recent paper
as a tractable alternative to general stochastic methods~\cite{Ozturk2019}.
Our framework was applied to an aircraft design problem with success, reducing
the probability of constraint violation of the design.

An interesting next step is to determine what role robustness plays in design geometry.
We hope to get insight into the degree to which
geometric sensitivities to design parameters are affected by the size of the uncertainty
sets we protect against. Searches on Google Scholar on topics such as `uncertain geometry',
`geometry sensitivity' and `robust geometry' yield no relevant results. As such, we
will be exploring an uncharted research area that has many potential applications in design.

\section{Background}

\subsection{Why optimization under uncertainty?}

Simply, we want to preserve constraint feasibility under perturbations of uncertain parameters,
with as little a penalty as possible to objective performance. In other words,
we want designs that protect against uncertainty least conservatively,
especially when compared to designs that leverage conventional methods
of uncertainty protection such as design with margins or multimission design.

Systems designed for the mean value of a parameter are doomed to be
sensitive to parametric uncertainty. In the simplest
example of a linear inequality (i.e. a hyperplane) separating the
feasible and infeasible set of designs, 50\% of designs would be
infeasible when parameters are perturbed.
In many dimensions this becomes an even more dismal proposition.
In an aircraft design example, we have shown that
designing for nominal values of uncertain parameters yields
design that are woefully inadequate in their ability to be robust
to uncertainty, with greater than 90\% probability of constraint
violation~\cite{Ozturk2019}.

Optimization under uncertainty introduces mathematical
rigor to design under uncertain parameters, and aims to
reduce the sensitivity of design performance to uncertain parameters, thereby reducing risk.

\subsection{Mathematical ideas behind robust optimization}

RO is a tractable method for optimization under uncertainty, and specifically under uncertain
parameters. It optimizes the worst-case objective outcome over uncertainty sets,
unlike general stochastic optimization methods which optimize statistics of the distribution
of the objective over probability distributions of uncertain parameters. As such, RO
sacrifices generality for tractability, probabilistic guarantees and engineering intuition. The following
descriptions have been paraphrased from~\cite{Ozturk2019}.

Given a general optimization problem under parametric uncertainty, we define the set of possible
realizations of uncertain vector of parameters $u$ in the uncertainty set $\mathcal{U}$. This
allows us to define the problem under uncertainty below.
\begin{align*}
    \text{min} &~~f_0(x) \\
    \text{s.t.}     &~~f_i(x,u) \leq 0,~\forall u \in \mathcal{U},~i = 1,\ldots,n
\end{align*}
This problem is infinite-dimensional, since it is possible to formulate an infinite number of constraints
with the countably infinite number of possible realizations of $u \in \mathcal{U}$. To circumvent this issue,
we can define the following robust formulation of the uncertain problem below.
\begin{align*}
    \text{min} &~~f_0(x) \\
    \text{s.t.}     &~~\underset{u \in \mathcal{U}}{\text{max}}~f_i(x,u) \leq 0,~i = 1,\ldots,n
\end{align*}
This formulation hedges against the worst-case realization of the uncertainty in the defined uncertainty
set. The set is often described by a norm, which contains possible uncertain outcomes from distributions with
bounded support
\begin{equation}
    \begin{split}
        \text{min} &~~f_0(x) \\
    \text{s.t.}     &~~\underset{u}{\text{max}}~f_i(x,u) \leq 0,~i = 1,\ldots,n \\
                    &~~\left\lVert u \right\rVert \leq \Gamma \\
        \end{split}
    \label{eq:normform}
\end{equation}
where $\Gamma$ is defined by the user as a global uncertainty bound. The larger the $\Gamma$,
the greater the size of the uncertainty set that is protected against.

\subsubsection{Clarification of nomenclature}

In the context of optimization, the inputs to a design problem are referred to as parameters,
and the outputs are the objective and design variables.
This can be confusing since ESP refers to its driving variables as parameters,
which are actually the design variables of the original problem.
In this case, optimization occurs prior to geometry generation,
so we refer to optimization inputs as parameters, optimization outputs (which drive geometry)
as design variables.

\section{Role of sensitivities}

As a result of the primal-dual interior point solution method,
GPkit returns the sensitivities of the objective function to each parameter, as shown below.
We use $F$, $x$ and $y$ to denote objectives, optimization parameters and design variables respectively.

\begin{equation}
\begin{split}
\frac{dF}{dx} &= \frac{\text{fractional change in objective}}{\text{fractional change in design parameter}} \\ 
\frac{dF}{dy} &= \frac{\text{fractional change in objective}}{\text{fractional change in variable}} = 0 \\ 
\end{split}
\end{equation}

The objective sensitivities to parameters are important since they signal
how much parameters affect the performance of the design relative to each other.
Since the solution of a \gls{gp} or \gls{sp} is at least locally optimal,
we expect that the variable sensitivity is 0. Note that the above are total derivatives.
$\frac{dF}{dx}$ means that other variables are free to change when the parameter is perturbed.
The partial derivative $\frac{\partial F}{\partial x}$ implies that other variables are fixed when the parameter is perturbed. 


\subsection{The Jacobian}

The total derivative of the objective w.r.t. parameters gives us the sensitivity of performance,
but not of geometry. To be able to determine the effects of parameters on some aspect of geometry,
such as OML, we require the matrix of total derivatives of design variables with respect to input parameters,
called the Jacobian. We paraphrase \cite{Martins2013} below to explain
the process of obtaining the Jacobian from total derivatives of the objective. 

\begin{equation}
\begin{split}
\frac{dF}{dx} &= \frac{\partial F}{\partial x} + \frac{\partial F}{\partial y} \frac{dy}{dx}  \\ 
\frac{dr}{dx} &= \frac{\partial R}{\partial x} + \frac{\partial R}{\partial y} \frac{dy}{dx} = 0
\end{split}
\end{equation}

The first equation above is simply a statement of the chain rule for 
the total derivative. The second equation
explores the idea that the total derivatives of the residuals of the optimization must be zero under small
perturbations of the parameters, meaning that the constraints must be satisfied.
We can rewrite this in the following format:
\begin{equation}
\frac{dy}{dx} = - \Big[\frac{\partial R}{\partial y} \Big]^{-1} \Big[ \frac{\partial R}{\partial x} \Big] 
\end{equation}

The adjoint analytic method from Gray et. al 
shows that one linear solve is required per objective and constraint variable. 
\begin{equation}
\begin{split}
\frac{dF}{dx} &= \frac{\partial F}{\partial x} + \Psi^T\Big[\frac{\partial R}{\partial x}\Big] \\ 
\Psi &= - \Big[\frac{\partial R}{\partial y}^T \Big]^{-1} \Big[\frac{\partial F}{\partial y}^T\Big]
\end{split}
\end{equation}

This means, for a given GP with $m$ parameters and $n$ variables, $m+n$ GP solves are required to generate the partial derivatives of the objective with respect to the parameters and variables, and then obtain the Jacobian. 

Given the solution of
our optimization problem, we can analytically compute the partial derivatives $\frac{\partial R}{\partial y}$
and $\frac{\partial R}{\partial x}$.
Geometry tools (eg. ESP) already have methods to handle the sensitivity of the geometry to design variables,
but these are not yet coupled to optimization tools like GPkit. By coupling the Jacobian of variable sensitivities
to parameters from GPkit with geometry sensitivities to design variables in ESP, we can hope to map how
robustness considerations affect geometry. 

\section{Goals}

The goals of this framework will be to:
\begin{itemize}
    \item \textbf{Determine the Jacobian of a solved \gls{gp}:}
                This will help us understand the local behavior
                of the design variables with respect to the input parameters.
    \item \textbf{Map variable sensitivities from the Jacobian to geometry for a sample problem:}
                For an aircraft design problem under zero uncertainty, we hope to be able to
                go from variables to geometry, and then map sensitivities onto the boundary representation.
    \item \textbf{Evaluate effect of robustness on geometry sensitivities:}
                Using the mapping, we hope to see how optimization under uncertainty
                helps reduce geometry sensitivity to design parameters. 
\end{itemize}

\bibliographystyle{new-aiaa}
\bibliography{main}

\end{document}

% - Release $Name:  $ -

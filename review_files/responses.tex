%%
%% Author: Berk
%% 1/27/2020
%%

% Preamble
\documentclass[11pt]{article}

% Packages
\usepackage{amsmath}
\usepackage{color}

% Document
\begin{document}

    \title{Responses to Referees: \\
    "Optimal Aircraft Design Decisions Under Uncertainty \\
    via Robust Signomial Programming"}
    
    \author{Berk \"Ozt\"urk, Ali Saab}

\maketitle

We thank the referees for their comments on the draft manuscript. Specific comments from the reviews (\textit{in italics}) are addressed below, and have been addressed {\color{blue} in blue} in the revised manuscript. 

\section{Reviewer 1}

\textit{This paper presents a method that transforms stochastic optimization problems to deterministic problems by considering the worst-case robust counterpart of each design constraint. The authors contribute a new formulation that leverages signomial programming to achieve efficient optimization under uncertainty. The paper is well written and the results demonstrate the contributions effectively.}

We thank the reviewer for their positive appraisal, and address their comments below. 

\textit{pg. 1: You should make the point that usually, optimization under uncertainty is in most cases untractable for more than a few variables due to computational cost. However, since you can solve signomial programming problems much faster than conventional formulations, you are able to solve relatively large optimization under uncertainty with a relatively low computational cost.}

The abstract has been updated to highlight the fact that RSPs are tractable and efficient. 

\textit{pg. 1: You should also emphasize the obvious caveat: The whole approach is limited to problems that can be described using signomial programming, which is by no means general.}

Abstract has been updated to emphasize the caveat: "provided that each constraint is SP-representable."

\textit{pg. 2: Remove bullets and convert to prose.}

The bullets have been reworded and converted to prose as suggested. 

\textit{pg. 3: The following references use a multimission/multipoint approach that go beyond the naive approach mentioned in the introduction. They are based on probability density functions and/or actual flight data and should be cited...}

All three references have been cited and added to the the literature on multimission/multipoint approaches on pg. 3. 
We have attempted to explain and contrast the approaches in the [Gallard 2013] and [Liem 2015,2017] papers
with our method. 

\textit{pg. 3: SO and RO are not well defined and their relation is not clear. 
An example of these two methods in a given application might help.}

On pg. 3, I have  clarified the contexts in which each method is used, 
and added a figure to illustrate the relationships between the parameter inputs
and objective functions of each problem. 

\textit{pg. 6-7: Let's also call the functions and parameters by their names in the context of optimization to improve clarity: 
"objective" f0, "constraints" fi, and "design variables" x. And make clear that the uncertain parameters are other parameters that in a deterministic problem would be fixed.}

I have referred to the functions in the context of optimization as suggested. Furthermore, 
a clarification has been added to Section II.A to explain the relationship between uncertain parameters and the 
deterministic problem, by explaining that fixed $u$ are recovered when $\mathcal{U}$ is a set with a single element. 

\textit{pg. 6: Wouldn't f0 be a function of u in general? It certainly is when you look at Table 1.}

The oversight has been corrected in all instances of design problem statement in Section II.A. 

\textit{pg. 7: Do you have issues with the max function being discontinuous and causing issues?}

This is not an issue in the particular implementation of the method
Since the method leverages existing robust linear programming (RLP)
methods, the maximum is a continuous and convex function. 
In fact, the inner maximization problem can be solved analytically, using conic
duality and the fact only parameters ($b$'s) in the robust problem are uncertain, 
as demonstrated by Saab et al. in Appendix A, Equation 29-35 of "Robust Designs via Geometric Programming", 
cited in the paper.
It is this fact that allows for the robust optimization problems to be solved deterministically. 
We did not feel that this insight warranted changes to the manuscript, since 
Saab et al. elaborate in detail about the RLP approximation of the GP. 

\textit{pg 14: "set bounds on parameters". By "parameters", you mean constraints, right? Be more precise.}

The reviewer was astute in noting that "bound" was the improper word to describe the $3\sigma$ values
defined by the designer. These $3\sigma$ values define the relative size
of the uncertainty set in each parameter, whether the set is a box or an ellipsoid.
To avoid confusion, we have modified "bound" to "size". 

\textit{pg. 15, Table 1: In my view, the airfoil thickness ratio would be a design variable.}

We agree with the reviewer, and have converted airfoil thickness to be a design variable. 
This has required the removal of airfoil thickness and wetted area ratio from the uncertain parameters
and the addition of two constraints to the model. The model definition has been
amended to reflect these changes. The modifications have significantly improved the model since 
airfoil thickness now is coupled to all relevant aircraft disciplines: aerodynamics, structures and 
fuel volume. This also required changing an example in Section II 
which explained that margin allocation for wing thickness 
is unintuitive, where air density was used as an example instead. 

\textit{pg. 15: A figure illustrating the types of uncertainty would be useful.}

Figure 4 has been added to illustrate the types of uncertainty considered, 
as well as the effects of changes in $\Gamma$ and $3\sigma$ more concretely. 

\textit{Fig. 4: All 3 fuel weight results should be plotted in on figure, and same for the probability of failure (in separate figure).}

TODO: improve Fig. 4. Will involve saving all results and remaking plots with the same settings. 

\textit{Conclusions: You have an insightful set of well expressed conclusions. You should remind the readers of a big caveat. The whole approach is limited to problems that can be described using signomial programming, which is by no means general.}

We thank the reviewer for their feedback, which has been critical for making improvements to the manuscript. We 
have revisited the paper to better define the scope of design problems that can be expressed as signomial 
programs, and addressed each comment to the best of our abilities. 

\section{Reviewer 2}

\textit{The paper presents an interesting and valuable addition to the toolset of stochastic optimization methods. 
The presentation style is clear and the mathematics seems to be correct.}

We thank the reviewer for their positive review. We hope that the reviewer
will agree with the revisions made in response to other reviews, since
they do not substantially change the conclusions of the paper
and improve its narrative. 

\end{document}

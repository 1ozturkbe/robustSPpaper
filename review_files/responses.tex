%%
%% Author: Berk
%% 1/27/2020
%%

% Preamble
\documentclass[11pt]{article}

% Packages
\usepackage{amsmath}

% Document
\begin{document}

    \title{Responses to Referees: \\
    "Optimal Aircraft Design Decisions Under Uncertainty \\
    via Robust Signomial Programming"}
    
    \author{Berk \"Ozt\"urk, Ali Saab}

\maketitle

We thank the referees for their comments on the draft manuscript. Specific comments from the reviews (\textit{in italics}) are addressed below. 

\section{Reviewer 1}

\textit{This paper presents a method that transforms stochastic optimization problems to deterministic problems by considering the worst-case robust counterpart of each design constraint. The authors contribute a new formulation that leverages signomial programming to achieve efficient optimization under uncertainty. The paper is well written and the results demonstrate the contributions effectively.}

\textit{pg. 1: You should make the point that usually, optimization under uncertainty is in most cases untractable for more than a few variables due to computational cost. However, since you can solve signomial programming problems much faster than conventional formulations, you are able to solve relatively large optimization under uncertainty with a relatively low computational cost.}

\textit{You should also emphasize the obvious caveat: The whole approach is limited to problems that can be described using signomial programming, which is by no means general.}

\textit{pg. 2: Remove bullets and convert to prose.}

\textit{pg. 3: The following references use a multimission/multipoint approach that go beyond the naive approach mentioned in the introduction. They are based on probability density functions and/or actual flight data and should be cited:}

* F. Gallard, M. Meaux, M. Montagnac, and B. Mohammadi. Aerodynamic aircraft design for mission performance by multipoint optimization. In 21st AIAA Computational Fluid Dynamics Conference. American Institute of Aeronautics and Astronautics, jun 2013. doi:10.2514/6.2013-2582.

* R. P. Liem, G. K. W. Kenway, and J. R. R. A. Martins. Multimission aircraft fuel burn minimization via multipoint aerostructural optimization. AIAA Journal, 53(1):104–122, January 2015. doi:10.2514/1.J052940.

* R. P. Liem, J. R. R. A. Martins, and G. K. Kenway. Expected drag minimization for aerodynamic design optimization based on aircraft operational data. Aerospace Science and Technology, 63:344–362, April 2017. doi:10.1016/j.ast.2017.01.006.


pg. 3: SO and RO are not well defined and their relation is not clear. An example of these two methods in a given application might help.

pg. 6-7: Let's also call the functions and parameters by their names in the context of optimization to improve clarity: "objective" f0, "constraints" fi, and "design variables" x. And make clear that the uncertain parameters are other parameters that in a deterministic problem would be fixed.

\textit{pg. 6: Wouldn't f0 be a function of u in general? It certainly is when you look at Table 1.}

\textit{pg. 7: Do you have issues with the max function being discontinuous and causing issueDid you consider using}

\textit{pg 14: "set bounds on parameters". By "parameters", you mean constraints, right? Be more precise.}

\textit{pg. 15, Table 1: In my view, the airfoil thickness ratio would be a design variable.}

\textit{pg. 15: A figure illustrating the types of uncertainty would be useful.}

\textit{Fig. 4: All 3 fuel weight results should be plotted in on figure, and same for the probability of failure (in separate figure).}

\textit{Conclusions: You have an insightful set of well expressed conclusions. You should remind the readers of a big caveat. The whole approach is limited to problems that can be described using signomial programming, which is by no means general.}


\section{Reviewer 2}

\textit{The paper presents an interesting and valuable addition to the toolset of stochastic optimization methods. The presentation style is clear and the mathematics seems to be correct.}

\end{document}

\section{Mathematical Background}

This section has been borrowed entirely from~\cite{Ozturk2019}. Please continue to

\subsection{Robust Optimization}

Given a general problem under parametric uncertainty, we can define the set of possible
realizations of uncertain vector of parameters $u$ in the uncertainty set $\mathcal{U}$. This
allows us to define the problem under uncertainty below.

\begin{align*}
    \text{minimize} &~f_0(x) \\
    \text{s.t.}     &~f_i(x,u) \leq 0,~\forall u \in \mathcal{U},~i = 1,\ldots,n \\
\end{align*}

This problem is infinite-dimensional, since it is possible to formulate an infinite number of constraints
with the countably infinite number of possible realizations of $u \in \mathcal{U}$. To circumvent this issue,
we can define the following robust formulation of the uncertain problem below.

\begin{align*}
    \text{minimize} &~f_0(x) \\
    \text{s.t.}     &~\underset{u \in \mathcal{U}}{\text{max}}~f_i(x,u) \leq 0,~i = 1,\ldots,n \\
\end{align*}

This formulation hedges against the worst-case realization of the uncertainty in the defined uncertainty
set. This is often posed by creating an uncertainty set to contain all possible
realizations of the uncertainty we are concerned about, usually through a norm,

\begin{align*}
    \label{eq:normform}
    \text{minimize} &~f_0(x) \\
    \text{s.t.}     &~\underset{u}{\text{max}}~f_i(x,u) \leq 0,~i = 1,\ldots,n \\
                    &~\norm{u} \leq \Gamma \\
\end{align*}

where $\Gamma$ is defined by the user as an uncertainty bound.

\subsection{Geometric Programming}
A \textbf{geometric program in posynomial form} is a log-convex optimization problem of the form:
\begin{equation}
\begin{aligned}
	& \text{minimize} && f_0 \left(\vec{u}\right) \\
	& \text{subject to} && f_i \left(\vec{u}\right) \leq 1, i = 1,...,m_p\\
	& && h_i \left(\vec{u}\right) = 1, i = 1, ...,m_e\\
\end{aligned}
\label{GP_standard}
\end{equation}
where each $f_i$ is a {\em posynomial}, each $h_i$ is a {\em monomial}, $m_p$ is the number of posynomials, and $m_e$ is the number of monomials. A monomial $h(\vec{u})$ is a function of the form:
\begin{displaymath}
	h_i(\vec{u}) = e^{b_i}\textstyle{\prod}_{j=1}^{n}{u_j}^{a_{ij}}
\end{displaymath}
where $a_{ij}$ is the $j^{th}$ component of a row vector $\vec{a_i}$ in $\mathbb{R}^n$, $u_j$ is the $j^{th}$ component of a column vector $\vec{u}$ in $\mathbb{R}^n_+$ , and $b_i$ is in $\mathbb{R}$. A posynomial $f(\vec{u})$ is the sum of $K \in \mathbb{Z}^+$ monomials:
\begin{displaymath}
	f_i(\vec{u}) = \textstyle{\sum_{k=1}^{K}}e^{b_{ikj}}\prod_{j=1}^{n}{u_j}^{a_{ikj}}
\end{displaymath}
where $a_{ikj}$ is the $j^{th}$ component of a row vector $\vec{a_{ik}}$ in $\mathbb{R}^n$, $u_j$ is the $j^{th}$ component of a column vector $\vec{u}$ in $\mathbb{R}^n_+$, and $b_{ik}$ is in $\mathbb{R}$ \cite{Boyd2007}.\\
A logarithmic change of the variables $x_j = \log(u_j)$ would turn a monomial into {\em  the exponential of an affine function} and a posynomial into {\em the sum of exponentials of affine functions}. A transformed monomial $h_i(\vec{x})$ is a function of the form:
\begin{displaymath}
    h_i(\vec{x}) = e^{\vec{a_i}\vec{x} + b_i}
\end{displaymath}
where $\vec{x}$ is a column vector in $\mathbb{R}^n$. A transformed posynomial $f_i(\vec{x})$ is the sum of $K_i \in \mathbb{Z}^+$ monomials:
\begin{displaymath}
    f_i(\vec{x}) = \textstyle{\sum_{k=1}^{K_i}}e^{\vec{a_{ik}}\vec{x} + b_{ik}}
\end{displaymath}
where $\vec{x}$ is a column vector in $\mathbb{R}^n$. A geometric program with transformed constraints is a \textbf{geometric program in exponential form}.

The positive nature of exponential functions restricts the space spanned by posynomials and limits the applications of \gls{gp}s to certain classes of problems. The limited applicability of \gls{gp}s has motivated the introduction of signomials.

\subsection{Signomial Programming}
A {\em signomial} can be defined as the difference between two posynomials, consequently, an SP is a non-log-convex optimization problem of the form:
\begin{equation}
\begin{aligned}
&\text{minimize } && f_{0}(\vec{x}) \\
&\text{subject to } && f_{i}(\vec{x}) - g_{i}(\vec{x})& \leq 0, i = 1, ...., m \\
\end{aligned}
\end{equation} 
where $f_{i}$ and $g_{i}$ are both posynomials, and $\vec{x}$ is a column vector in $\mathbb{R}^n$. 

Reliably solving an SP to a local optimum has been described in \cite{Boyd2007} and \cite{Lipp2016}.
A common solution heuristic involves solving an SP as a sequence of GPs, where each GP is a local approximation of the SP.
Although it is a powerful tool, applications involving SPs are usually prone to uncertainties that have a significant effect on the solution.

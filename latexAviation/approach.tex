\section{Approach to Solving \gls{rsp}s}

This section presents a heuristic algorithm to safely solve a \gls{rsp}
based on our previous discussion on robust geometric programming.

\subsection{General RSP Solver}
As mentioned before, a common heuristic algorithm to solve an SP is
by sequentially solving local \gls{gp} approximations, but the solution is not guaranteed
to be globally optimal. Our approach to solve an RSP is based on sequentially solving
local \gls{rgp} approximations. Below we provide a step-by-step algorithm to solve an \gls{rsp}:

\begin{enumerate}
    \item Choose an initial guess $x_0$
    \item Repeat
    \begin{enumerate}
        \item Find the local GP approximation of the SP at $x_i$.
        \item Find the RGP formulation of the GP.
        \item Solve the RGP to obtain $x_{i+1}$.
        \item If $x_{i+1} \approx x_{i}$: break
    \end{enumerate}
\end{enumerate}

Similar to an SP, a good initial guess would lead to faster convergence and possibly a better solution.
A quick candidate is the deterministic solution of the uncertain SP, which will certainly lead to a faster convergence.

\begin{figure}
    \begin{center}
    \begin{tikzpicture}[auto, align=center, text width=2.75cm, scale = 0.9]
        \begin{scope}[node distance=2cm]
        \node[block, name=detSP] at (0,0) (detSP) {Solve deterministic SP};
        \node[block, name=localGP] at (4,0) (localGP) {Make local GP approximation};
        \node[block, name=localRGP] at (10,0) (localRGP) {Formulate local RGP};
        \node[block, name=solveRGP] at (14,0) (solveRGP) {Solve local RGP};
        \node[block, name=xi] at (14,-2) (xi) {$\Delta x \leq reltol$?};
        \node[block, name=solution] at (14,-4) (solution) {Solution $x_{n}$};

        \draw[->] (detSP) -- node[name=x0] {$x_0$} (localGP);
        \draw[->] (localGP) -- node[name=choosemethod] {Choose methodology} (localRGP);
        \draw[->] (localRGP) -- (solveRGP);
        \draw[->] (solveRGP) -- (xi);
        \draw[->] (xi) -| node[name=no] {No, $x_{i+1} = x_i.$} (localGP);
        \draw[->] (xi) -- node[name=yes] {Yes.} (solution);
        \end{scope}
    \end{tikzpicture}
    \caption{A block diagram showing the steps of solving an RSP.}
        \label{fig:rspsolve}
\end{center}
\end{figure}

Any of the previously mentioned methodologies can be used to formulate the local RGP approximation. 
However, depending on the \gls{rgp} formulation chosen to solve an \gls{rsp}, the last formulation and solution
blocks in Figure \ref{fig:rspsolve} are tweaked slightly for a faster convergence.
The two main methods used in this paper are the Best Pairs and the Linearized Perturbations
methods proposed by~\cite{Ozturk2019}.

% Aero Astro template
% Max Opgenoord
% August 2017

% ============================== Tikz ================================== %
% ------------------------------ Packages ------------------------------ %
\usepackage{pgf,pgfpages,pgfplots}
\usepackage{tikz-3dplot}
\usepackage{tkz-euclide}
\usepackage{adjustbox} % Allow for scaling of blocks (especially useful
                       % for tikzpictures)
% ---------------------------------------------------------------------- %

% ------------------------------ Formatting ---------------------------- %
\usetikzlibrary{arrows,shapes,positioning,shadows,trees,patterns,intersections,matrix,fit,backgrounds,automata,calc}
\pgfplotsset{compat=newest}
  \pgfplotsset{plot coordinates/math parser=false}
\newlength\figureheight
    \newlength\figurewidth
\usetkzobj{all}
\tikzset{
basic box/.style = {
    shape = rectangle,
    align = center,
    draw  = #1,
    fill  = #1!25,
    rounded corners},
}
\tikzset{
  invisible/.style={opacity=0},
  visible on/.style={alt={#1{}{invisible}}},
  alt/.code args={<#1>#2#3}{%
    \alt<#1>{\pgfkeysalso{#2}}{\pgfkeysalso{#3}}
  },
}
\tikzset{onslide/.code args={<#1>#2}{%
  \only<#1>{\pgfkeysalso{#2}} % \pgfkeysalso doesn't change the path
}}
% ---------------------------------------------------------------------- %

% ------------------------------ 3D Plots ------------------------------ %
\makeatletter
\tikzoption{canvas is xy plane at z}[]{%
  \def\tikz@plane@origin{\pgfpointxyz{0}{0}{#1}}%
  \def\tikz@plane@x{\pgfpointxyz{1}{0}{#1}}%
  \def\tikz@plane@y{\pgfpointxyz{0}{1}{#1}}%
  \tikz@canvas@is@plane
}
\makeatother
% ---------------------------------------------------------------------- %
% ====================================================================== %

% ============================== Add extra commands ==================== %
% ------------------------------ Misc ---------------------------------- %
\newcommand{\bomega}{\textrm{\boldmath$\omega$}}
\DeclareMathOperator*{\argmin}{arg\,min}
% ---------------------------------------------------------------------- %

% ------------------------------ Partials ------------------------------ %
\newcommand{\PDer}[2]{%
    \frac{\partial #1}{\partial #2}
}
\newcommand{\PDerT}[2]{%
    \frac{\partial^2 #1}{\partial #2^2}
}
\newcommand{\PDerF}[2]{%
    \frac{\partial^4 #1}{\partial #2^4}
}
\newcommand{\PDerFM}[3]{%
    \frac{\partial^4 #1}{\partial #2^2 \partial #3^2}
}
\newcommand{\PDerTh}[2]{%
    \frac{\partial^3 #1}{\partial #2^3}
}
\newcommand{\PDerThM}[3]{%
    \frac{\partial^3 #1}{\partial #2 \partial #3}
}

\newcommand{\PDerM}[3]{%
    \frac{\partial^2 #1}{\partial #2 \partial #3}
}
\newcommand{\DerF}[2]{%
    \frac{\mathrm{d}^4 #1}{\mathrm{d} #2^4}
}
\newcommand{\Der}[2]{%
    \frac{\mathrm{d} #1}{\mathrm{d} #2}
}
\newcommand{\DerT}[2]{%
    \frac{\mathrm{d}^2 #1}{\mathrm{d} #2^2}
}
\newcommand{\DerTh}[2]{%
    \frac{\mathrm{d}^3 #1}{\mathrm{d} #2^3}
}
% ---------------------------------------------------------------------- %
% ====================================================================== %
